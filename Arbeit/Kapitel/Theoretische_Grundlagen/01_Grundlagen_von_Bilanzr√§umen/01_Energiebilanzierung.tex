\subsection{Energiebilanzierung}
\subsubsection{Definition}
Die Definition des Konzepts der Bilanzierung hängt von der Perspektive ab. Aus verfahrenstechnischer Sicht wird die Bilanzierung nach 
Rönsch (2015, S. 87) in drei Bilanzgleichungen unterteilt: die Massenbilanz, die Energiebilanz und die Impulsbilanz. 
Zur Beantwortung dieser Forschungsfrage hat insbesondere die Energiebilanz eine hohe Relevanz.

Die Energiebilanz beruht auf dem Energieerhaltungssatz (\cite[S. 87]{Rönsch.2015}), der das Prinzip der Erhaltung 
der Energie ausdrückt (\cite[S. 57]{Baehr.1966}). Der Energieerhaltungssatz bezieht sich auf alle Erscheinungsformen, in denen Energie auftritt, 
und besagt, dass es unmöglich ist, Energie zu erzeugen oder zu vernichten (\cite[S. 57]{Baehr.1966}). 
Für zu bilanzierende Systeme bedeutet dies, dass die Energie in einem abgeschlossenen, adiabaten System über die Zeit 
konstant ist (\cite[S. 87]{Rönsch.2015}). 
Adiabat bedeutet in diesem Kontext, dass das System keinen Wärmeaustausch mit der Umgebung hat (\cite[S. 87]{Rönsch.2015}). 

Für Systeme, die in der Lage sind, Energie zu speichern, impliziert dies nach Rönsch (2015, S. 87f.), 
dass die darin gespeicherte Energie gleich der Differenz aus ein- und austretenden Energieströmen ist. 
Für offene, nicht-adiabate Systeme ohne Speicherfähigkeit gilt, dass die Differenz der ein- und austretenden Energieströme Null ist 
(\cite[S. 87f.]{Rönsch.2015}).
Das von Rönsch (2015, S. 87f.) beschriebene Verhalten eines Systems bezüglich der Energiespeicherung, lässt sich mathematisch 
vereinfacht mit der Gleichung \eqref{energiebilanzierungsgleichung_Rönsch} darstellen:

\begin{equation}
E_{\text{gespeichert}} = \sum E_{\text{eingang}} - \sum E_{\text{ausgang}}
\label{energiebilanzierungsgleichung_Rönsch}
\end{equation}

\begin{description}
    \item \(E_{\text{gespeichert}}\): Im System gespeicherte Energie.
    \item \(E_{\text{eingang}}\): Energie eines eintretenden Energiestroms.
    \item \(E_{\text{ausgang}}\): Energie eines austretenden Energiestroms.
    \item Für offene, nicht-adiabate Systeme ohne Energiespeicher gilt:
    \[
    E_{\text{gespeichert}} = 0
    \]
    \item In diesem Fall ist die zugeführte Energie gleich der abgegebenen Energie:
    \[
    \sum E_{\text{eingang}} = \sum E_{\text{ausgang}}
    \]
\end{description}

Diese Gleichung beschreibt einen allgemeinen Ansatz zur Energiebilanzierung der im System gespeicherten Energie. 
Im Rahmen der Bilanzierung komplexer Systeme kann jedoch eine detailliertere Bilanzierung einzelner Zustandsgrößen im System erforderlich sein.

Dieses Problem wird von der von Ahrendts (2014, Kapitel 1.5) aufgestellten Bilanzgleichung im Kontext der Thermodynamik addressiert. 
Die Gleichung basiert auf dem Fakt, dass sich für jede mengenartige Zustandsgröße, die über die Grenze eines Systems transportiert wird, eine 
Bilanz aufstellen lässt (\cite[Kapitel 1.5]{Ahrendts.2014}). 
Diese Bilanz umfasst ein- und austretende Ströme sowie im System enthaltene Energiequellen und -senken und ermittelt die 
Geschwindigkeit der Änderung des Bestands der zu bilanzierenden Zustandsgröße im System (\cite[Kapitel 1.5]{Ahrendts.2014}).

Die von Ahrendts (2014, Kapitel 1.5) aufgestellte Bilanzgleichung wird in den Formeln \eqref{BilanzierungsgleichungAhrendt} und 
\eqref{BilanzierungsgleichungAhrendtStrom} dargestellt.

\begin{equation}
    dX_{\text{j}}/d\tau = (\sum \dot{X}_{\text{j,e}} - \sum \dot{X}_{\text{j,a}}) + (\dot{X}_{\text{j,Quell}} - \dot{X}_{\text{j,Senk}})
    \label{BilanzierungsgleichungAhrendt}
\end{equation}

\begin{description}
    \item \(X_{\text{j}}\): Zustandsgröße.
    \item \(X_{\text{j,e}}\): Über die Systemgrenze zufließende Zustandsgröße.
    \item \(X_{\text{j,a}}\): Über die Systemgrenze abfließende Zustandsgröße.
    \item \(X_{\text{j,Quell}}\): Quellen der Zustandsgröße im System.
    \item \(X_{\text{j,Senk}}\): Senken der Zustandsgröße im System.
\end{description}

Im Rahmen der Formel \eqref{BilanzierungsgleichungAhrendt} wird der Strom einer Zustandsgröße \(X_{\text{j}}\) in Gleichung 
\eqref{BilanzierungsgleichungAhrendtStrom} definiert.

\begin{equation}
    \dot{X}_{\text{j}} = \lim_{\Delta\tau \to 0} \Delta X_{\text{j}}/ \Delta\tau
    \label{BilanzierungsgleichungAhrendtStrom}
\end{equation}

\begin{description}
    \item \(X_{\text{j}}\): Zustandsgröße.
    \item \(\Delta X_{\text{j}}\): Menge der Größe \(X_{\text{j}}\) im Zeitintervall \(\Delta \tau\).
    \item \(\Delta \tau\): Zeitintervall.
\end{description}

Die Gleichung \eqref{BilanzierungsgleichungAhrendt} in Verbindung mit \eqref{BilanzierungsgleichungAhrendtStrom} beschreibt die Geschwindigkeit der Änderung des 
Bestands der Größe \(X_{\text{j}}\) als Summe der Differenzen zwischen den über die Systemgrenze zu- und abfließenden Strömen der Zustandsgröße 
\(X_{\text{j}}\) sowie den Quell- und Senkenströmen der Größe \(X_{\text{j}}\) innerhalb des Systems. 

Die Gleichungen \eqref{energiebilanzierungsgleichung_Rönsch} und \eqref{BilanzierungsgleichungAhrendt} mit \eqref{BilanzierungsgleichungAhrendtStrom} 
formulieren eine grundlegende und vereinfachte mathematische Beschreibung der Bilanzierungsrechnung im Kontext der Thermodynamik und Verfahrenstechnik. 
Sie beschreiben jedoch die grundlegende Struktur einer Bilanz. 
Die in \eqref{BilanzierungsgleichungAhrendt} beschriebene Zustandsgröße stellt die zu bilanzierende Größe dar und hat somit eine zentrale Rolle in der 
Bilanzierungsrechnung. In \eqref{energiebilanzierungsgleichung_Rönsch} wird die Gesamtenergie eines Systems als Zustandsgröße betrachtet, 
und es wird ein Zusammenhang zwischen der Zustandsgröße und der thermodynamischen Klassifikation der Speicherfähigkeit eines Systems hergestellt.
Die Größe und Veränderung einer Zustandsgröße innerhalb eines Systems ist nach der Definition von Ahrendts (2014, Kapitel 1.5) zum einen  
von den ein- und austretenden Zustandsströmen der Zustandsgröße abhängig. 
Des Weiteren ist die Zustandsgröße von Quellen und Senken der Zustandsgröße innerhalb des Systems abhängig.
Das System stellt den Kontext der Bilanz dar, indem es die Grenzen definiert und sich gemäß \eqref{energiebilanzierungsgleichung_Rönsch} auf das Verhalten der 
Zustandsgröße auswirkt. Nach \eqref{BilanzierungsgleichungAhrendt} können in einem System mehrere Zustandsgrößen bilanziert werden.

Im Folgenden werden die aus \eqref{energiebilanzierungsgleichung_Rönsch} und \eqref{BilanzierungsgleichungAhrendt} mit \eqref{BilanzierungsgleichungAhrendtStrom} 
abgeleiteten Bestandteile einer Bilanz im Anwendungskontext des Problemraums analysiert. 

\subsubsection{Zustandsgrößen in der Energiebilanzierung}

% Wie kann zeitliche Aggregation in die Definition integriert werden?

% Was ist ein Strom?

% Detaillierter beschreiben und Schlussfolgerungen ziehen.

\subsubsection{Energieströme}
% TODO: Beleg - in der Energiebilanzierung werden Energieströme betrachtet.

% Wie realisiert man Ströme (Zeitreihe -> Strom)?
% Praxisbeispiele.
% (praktische Herausforderungen).

\subsubsection{Energiequellen und -senken}
% Was charakterisiert Energiequellen und -senken?
% Praxisbeispiele.
% (praktische Herausforderungen).
