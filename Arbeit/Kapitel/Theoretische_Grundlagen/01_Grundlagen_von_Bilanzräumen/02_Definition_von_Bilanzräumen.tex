\subsection{Konzept: Bilanzraum}

%%%%%%%%%%%%%%%%%%%%%%%%%%%%%%%%%%%%%%%%%%%%%%%%%%%%%%%%%%%%%%%%%%%%%%%%%%%%%%%%%%%%%%%%%%%%%%%%%%%%%%%%%%%%%%%%%%%%%%%%%%%%%%%%%%%%%%%%%%%%%%%%%%%%%%%%%%%%%%%%%%%

\subsubsection{Bilanzraumgrenzen zur Abbildung des bilanzierten Systems}
Eine Bilanz bezieht sich gemäß Ahrendts (2014, Kapitel 1.5) auf das von der Systemgrenze eingeschlossene Kontrollgebiet. 
Die Systemgrenze kann dabei unter Berücksichtigung der Zweckmäßigkeit frei definiert werden (\cite[Kapitel 1.5]{Ahrendts.2014}).
Das definierte System kann auch als Bilanzraum bezeichnet werden, da die Berechnung der in einen Bilanzraum ein- und austretenden Ströme als Bilanzierung bezeichnet wird (\cite[S. 65]{Rönsch.2015}).
Einen Ansatz zur Definition von Bilanzräumen liefert Miller (2016, S. 105) mit der Konkretisierung von Bewertungsräumen mittels Kriterien der Bilanzgrenze, dem 
Aggregationsniveau und der Bewertungseinheit. Das definierte System Dient der Bewertung der Nutzung der Ressourcen wobei der Effizienzbegriff eine zentrale Rolle spielt (\cite[S. 107]{Miller.2016}).
Miller (2016, S. 107) definiert die Effizienz nach Gleichung \eqref{EffizienzgleichungMiller}.

\begin{equation}
    \text{Effizienz} := \frac{\text{Erreichter Nutzen}}{\text{Aufwand}}
    \label{EffizienzgleichungMiller}
\end{equation}

Der Aufwand umfasst nach Miller (2016, S. 108f.) unterschiedliche Ressourcen, wobei im Kontext energiewirtschaftlicher Fragestellungen der Fokus auf der Ressource Energie 
liegt. Der Nutzen ist vom Untersuchungsgegenstand, also der zu bilanzierenden Zustandsgröße, abhängig und wird im Kontext der energiewirtschaftlichen 
Fragestellung häufig über Energiedienstleitstungen operationalisiert (\cite[S. 107]{Miller.2016}). Betrachtet man dass der Nutzen grundsätzlich durch Befriedigung 
von Bedürfnissen beschrieben wird entsteht im Kontext der Energiewirtschaft ein Nutzenergiebedarf zur Befriedigung der Bedürfnisse im Rahmen einer Energiedienstleitstung 
(\cite[S. 107]{Miller.2016}). Der Nutzen wird meist implizit durch den gewählten Untersuchungsgegenstand, also der bilanzierten Zustandsgröße, definiert (\cite[S. 110]{Miller.2016}).
Die in \eqref{EffizienzgleichungMiller} aufgestellte Nutzen-Aufwand Relation stellt die Grundlage der Definition der Bilanzraumgrenze dar. 
Die Bilanzraumgrenze lässt sich somit in die Aufwandsseitige Bilanzgrenze, die alle zu bilanzierenden Ressourcen umfasst und die Nutzenseitige Bilanzrgrenze die sich auf 
die zu Bilanzierende Energiedienstleitstung bezieht (\cite[S. 111]{Miller.2016}).  

Das von Miller (2016) beschriebene Konzept bringt eine neue Perspektive auf die in \eqref{BilanzierungsgleichungAhrendt} und \eqref{BilanzierungsgleichungAhrendtStrom} 
aufgestellte Bilanzgleichung. Sie bringt das Prinzip der Effizienz ein und teilt eine Bilanz in Aufwands und Nutzenseite. 
Aufwandsseitig sind die in \eqref{BilanzierungsgleichungAhrendt} zufließenden Ströme und Quellen der Zustandsgröße zu betrachten. 
Nutzenseitig müssen abfließende Ströme und Senken der Zustandsgröße betrachtet werden. Im Zentrum der Bilanzierung steht der Untersuchungsgegenstand, welcher in 
\eqref{BilanzierungsgleichungAhrendt} als Zustandsgröße definiert ist und einen Nutzen im Rahmen einer Energiedienstleitstung impliziert. 

%%%%%%%%%%%%%%%%%%%%%%%%%%%%%%%%%%%%%%%%%%%%%%%%%%%%%%%%%%%%%%%%%%%%%%%%%%%%%%%%%%%%%%%%%%%%%%%%%%%%%%%%%%%%%%%%%%%%%%%%%%%%%%%%%%%%%%%%%%%%%%%%%%%%%%%%%%%%%%%%%%%

\subsubsection{Zustandsgrößen in Bilanzräumen}
Im Rahmen der in \eqref{energiebilanzierungsgleichung_Rönsch}, \eqref{BilanzierungsgleichungAhrendt} und \eqref{BilanzierungsgleichungAhrendtStrom} beschriebenen Bilanzierung 
spielt die Zustandsgröße eine zentrale Rolle. 
Die Definition der Systemgrenze wird vom Untersuchungsgegenstand, also der Zustandsgrößen beeinflusst (\cite[S. 109]{Miller.2016}). Somit gilt die Zweckmäßigkeit der Systemgrenze 
auch für die zu untersuchenden Zustandsgrößen.

Die DIN EN ISO 50001:2018-12 gibt mit dem Ziel der fortlaufenden Verbesserung der energiebezogenen Leistung eine Vorgabe zur zweckmäßigen Definition der Zustandsgröße 
(\cite[S. 11]{DIN50001.2018}).
Da sich der Anwendungskontext der Bilanzierung auf energetische Größen im Bereich der Gebäudeenergie bezieht, ist eine zweckmäßige Auswahl der Zustandsgrößen zur Bewertung 
relevanter energetischer Parameter erforderlich.
Die Vornorm DIN V 18599-1:2018-09, herausgegeben vom Deutschen Institut für Normung e. V. (2018, S. 1), behandelt die energetische Bewertung von Gebäuden und stellt ein 
Verfahren zur Durchführung der Gesamtenergiebilanz bereit (\cite[S. 9]{DIN18599.2018}). Ihre Ausrichtung auf die energetische Bewertung erfüllt die Zweckmäßigkeit der 
DIN EN ISO 50001:2018-12. Der Fokus auf die Bewertung von Gebäuden entspricht der Zielsetzung der DIN V 18599-1:2018-09. 
Die Vorgaben der DIN V 18599-1:2018-09 bieten daher einen wertvollen Beitrag zur zielgerichteten Erarbeitung von zweckmäßigen Zustandsgrößen und werden im Folgenden 
näher untersucht.

Im Rahmen der energetischen bewertung von Gebäuden betrachtet die DIN V 18599-1:2018-09 die Bilanzierung des Nutz-, End- und Primärenergiebedarf (\cite{DIN18599.2018}).
In einem energiewirtschaftlichen Rahmen hat besonders der Nutzeenergiebedarf eine Große Bedeutung da er aus der Befriedigung der Bedürfnisse im Rahmen einer 
Energiedienstleistung resultiert (\cite[S. 107]{Miller.2016}).

\begin{figure}[H]
    \centering
    \includegraphics[width=0.8\textwidth]{../../Ressourcen/Bilder/Energiefluss_Miller.jpg}
    \caption{Energieflussschema. (Dargestellt von Miller (2016))}
    \label{fig:Energieflussschema_Miller}
\end{figure}

Im Energieflussdiagramm \eqref{fig:Energieflussschema_Miller} wird die Nutzenergie zwischen der Endenergie und der Energiedienstleistungen verortet. 
Da die Endenergie die Energiemenge ist, die dem Bilanzraum zur bestimmungsgemäßen Nutzung bereitgestellt wird (\cite[Kapitel 3.1.2]{DIN18599.2018}), 
Repräsentiert diese Energieform die Menge der potentiellen Ressourcen auf Aufwandsseite der Bilanzierung.
Außerdem wird der praktische Sachverhalt veranschaulicht dass bei der Umwandlung von Endenergie in Energiedienstleistungen über die Nutzenergie 
Umwandlungs und Transportverluste berücksichtigt werden müssen.
Dass Zustandsvariablen den Nutzen, der durch Energiedienstleistungen operationalisiert wird, impliziert (\cite[S. 110]{Miller.2016}), zeigt die Notwendigkeit zur 
festlegung geeigneter Zustandsvariablen des Nutzenergiebedarfs im Anwendungskontext.

Die DIN V 18599-1:2018-09 definiert den Nutzenergiebedarf für Heizung, Kühlung, Lüftung, Trinkwarmwasser und Beleuchtung als relevante Untersuchungsgegenstände 
zur energetischen Bewertung von Gebäuden (\cite{DIN18599.2018}). 
Somit versteht sich der Nutzenergiebedarf im Rahmen der Vornorm als Überbegriff für Nutzwärmebedarf, Nutzkältebedarf, Nutzenergiebedarf für Tinkwarmwasser, Beleuchtung und 
Befeuchtung definiert (\cite[Kapitel 3.1.3]{DIN18599.2018}).
Die Nutzeenergie für die Beleuchtung versteht sich als die Energiemenge die zur Ausreichenden Beleuchtung der Gebäudezone aufgewendet werden muss (\cite[Kapitel 5.3.1]{DIN18599.2018}).
Der Nutzwärmebedarf hingegen ist die Wämemenge, die der Gebäudezone zusätzlich (bedarfs-)geregelt zugeführt werden muss, um die vorgegebene 
Sollinnentemperatur einzuhalten (\cite[Kapitel 5.3.1]{DIN18599.2018}).
Im Rahmen der Luftaufbereitung versteht sich die Nutzenergie als Energiemenge die zum Erwärmen, Kühlen, Befeuchten und Entfeuchten der Luft in einer 
raumlufttechnischen Anlage zu- beziehungsweise abgeführt werden muss um den erforderlichen Zuluftzustand zu erreichen (\cite[Kapitel 5.3.1]{DIN18599.2018}).
Die Nutzenergie für die Trinkwarmwasserbereitung bedeutet die Energiemenge für die Erwärmung des Trinkwassers von der Kaltwassertemperatur auf die 
Warmwassertemperatur an der Entnahmestelle (\cite[Kapitel 5.3.1]{DIN18599.2018}).

%%%%%%%%%%%%%%%%%%%%%%%%%%%%%%%%%%%%%%%%%%%%%%%%%%%%%%%%%%%%%%%%%%%%%%%%%%%%%%%%%%%%%%%%%%%%%%%%%%%%%%%%%%%%%%%%%%%%%%%%%%%%%%%%%%%%%%%%%%%%%%%%%%%%%%%%%%%%%%%%%%%

\subsubsection{Energieströme in Bilanzräumen}
Die Veränderung einer Zustandsgröße innerhalb eines Systems ist nach der von Ahrendts (2014, Kapitel 1.5) aufgestellten Gleichung zum einen  
von den ein- und austretenden Zustandsströmen der Zustandsgröße abhängig. Dazu beitragend wird in der Gleichung \eqref{BilanzierungsgleichungAhrendtStrom}
der Strom eines Zustands als Menge der Zustandsgröße in einem infinitesimal kleinen Zeitintervall definiert, welches im Grenzwert gegen 0 geht. 
Die Definition von Ahrendts (2014, Kapitel 1.5) gibt also Auskunft über die Veränderung der Zustandsgröße zu einem spezifischen Zeitpunkt.

Der in der DIN EN ISO 50001:2018-12 geforderte Nachweis der fortlaufenden Verbesserung der Energiebezogenen Leistung zeigt jedoch eine weitere notwendige Perspektive 
auf den Sachverhalt der Bilanzierung von Zustandsgrößen. Und zwar meint "Fortlaufend" laut DIN EN ISO 50001:2018-12 das Auftreten über einen Zeitraum (\cite[S. 38]{DIN50001.2018}). 
Das impliziert die Notwendigkeit die Bilanzierung einer Zustandsgröße auch in einem frei definierbaren Zeitintervall zu betrachten. 

% TODO: Recherche neue Gleichung
% Wie kommt man von der gegebenen Gleichung (Veränderung der Zustandsgröße an einem Zeitpunkt) zu einer Gleichung die die Größe der Zustandsgröße in einem Festen Zeitintervall beschreibt (Strom -> Zeitreihe)?

\subsubsection{Energiequellen und -senken in Bilanzräumen}
Gleichung \eqref{BilanzierungsgleichungAhrendt} unterscheidet zwmischen Quellen, Senken und zu- beziehungsweise abfließenden Strömen der Zustandsgröße. 
Quell- und Senkenströme treten in einer Energiebilanz nach dem ersten Hauptsatz der Thermodynamik nicht auf da Energie als Erhaltungsgröße ist (\cite[S. 14]{Ahrendts.2014}). 
Im Rahmen der DIN EN ISO 50001:2018-12 bezieht sich der Begriff Energie auf verschiedene Arten von Energie die erworben, gespeichert, aufbereitet, in einer Einrichtung oder einem Prozess verwendet 
oder zurückgewonnen werden können (\cite[Kapitel 3.5.1]{DIN50001.2018}). Energie kann im Rahmen der Norm als Elektrizität, Brennstoff, Dampf, Wärme, Druckluft oder vergleichbares Medium auftreten 
(\cite[Kapitel 3.5.1]{DIN50001.2018}).
Folglich werden alle Energieströme, in denen Energie in eine Energieform umgewandelt wird, die nicht die genannten Kirterien erfüllen als Energiesenken betrachtet. 
Analog dazu werden alle Energieströme, bei denen Energie, die nicht den von der Norm aufgestellten Kriterien entspricht, in eine nach ISO 50001 definierte Energieform 
umgewandelt wird, als Energiequellen betrachtet.




% TODO: Definition Haupmerkmale, Kontext SEU-Bilanz
% TODO: Umformulieren

% Bei der Wahl einer Zustandsgröße haben neben der Zweckmäßigkeit auch die Messinfrastruktur in Organisationen eine Relevanz.
% Denn die Gleichung \eqref{BilanzierungsgleichungAhrendt} beschreibt das Verhalten der Zustandsgröße in Abhängigkeit von den zu- und abfließenden Zustandsgrößen und den 
% Quellen und Senken der Zustandsgröße. Wenn die Ströme, Quellen und Senken der Zustandsgröße nicht vollständig Erfasst werden ist somit auch keine korrekte Bilanzierung 
% möglich. 
% Die DIN EN ISO 50001:2018-12 (2018, Kapitel 6.6, A.6.6) stellt Anforderungen und Qualitätskriterien an die Datensammlung in Organisationen.
% Diese verpflichtet Organisationen dazu Hauptmerkmale ihrer Tätigkeiten, die sich auf die energiebezogene Leistung auswirken zu identifizieren, und diese in geplanten 
% Zeitabständen zu messen, überwachen und analysieren (\cite[S. 23]{DIN50001.2018}).
% Teil der zu erfassenden Hauptmerkmale sind relevante Variablen bezüglich wesentlicher Energieeinsätze, den Energieverbrauch bezüglich wesentlicher Einsätze 
% und der Organisation und betriebliche Kriterien bezüglich wesentlicher Energieeinsätze(\cite[S. 23]{DIN50001.2018}).
% Die komplexität der Umsetzung ist dabei nicht vorgeschrieben und kann von einfachen Zählwerten bis hin zu umfangreichen Werten aus Überwachungs- und Messystemen mit 
% Softwareandwendung reichen (\cite[S. 36]{DIN50001.2018}).