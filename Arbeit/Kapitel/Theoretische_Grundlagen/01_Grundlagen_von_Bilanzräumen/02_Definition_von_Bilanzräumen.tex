\subsection{Definition von Bilanzräumen}

% TODO: Definition Haupmerkmale, Kontext SEU-Bilanz
% TODO: Umformulieren

Bei der Wahl einer Zustandsgröße haben neben der Zweckmäßigkeit auch die Messinfrastruktur in Organisationen eine Relevanz.
Denn die Gleichung \eqref{BilanzierungsgleichungAhrendt} beschreibt das Verhalten der Zustandsgröße in Abhängigkeit von den zu- und abfließenden Zustandsgrößen und den 
Quellen und Senken der Zustandsgröße. Wenn die Ströme, Quellen und Senken der Zustandsgröße nicht vollständig Erfasst werden ist somit auch keine korrekte Bilanzierung 
möglich. 
Die DIN EN ISO 50001:2018-12 (2018, Kapitel 6.6, A.6.6) stellt Anforderungen und Qualitätskriterien an die Datensammlung in Organisationen.
Diese verpflichtet Organisationen dazu Hauptmerkmale ihrer Tätigkeiten, die sich auf die energiebezogene Leistung auswirken zu identifizieren, und diese in geplanten 
Zeitabständen zu messen, überwachen und analysieren (\cite[S. 23]{DIN50001.2018}).
Teil der zu erfassenden Hauptmerkmale sind relevante Variablen bezüglich wesentlicher Energieeinsätze, den Energieverbrauch bezüglich wesentlicher Einsätze 
und der Organisation und betriebliche Kriterien bezüglich wesentlicher Energieeinsätze(\cite[S. 23]{DIN50001.2018}).
Die komplexität der Umsetzung ist dabei nicht vorgeschrieben und kann von einfachen Zählwerten bis hin zu umfangreichen Werten aus Überwachungs- und Messystemen mit 
Softwareandwendung reichen (\cite[S. 36]{DIN50001.2018}).