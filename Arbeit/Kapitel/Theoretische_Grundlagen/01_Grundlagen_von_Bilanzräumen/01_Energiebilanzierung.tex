\subsection{Energiebilanzierung}
\subsubsection{Energiebilanzierung aus Perspektive der Verfahrenstechnik}
Die Definition des Konzepts Bilanzierung hängt von der Perspektive ab. So wird die Bilanzierung aus Verfahrenstechnischer Perspektive nach 
Rönsch (2015, S. 87) in drei Bilanzgleichungen aufgegliedert: der Massenbilanz, der Energiebilanz und der Impulsbilanz.
Die Energiebilanz beruht dabei auf dem Energieerhaltungssatz (\cite[S. 87]{Rönsch.2015}), welcher das Prinzip von der Erhaltung
der Energie ausdrückt (\cite[S. 57]{Baehr.1966}). Der Energieerhaltungssatz bezieht sich auf alle Erscheinungsformen in denen Energie auftritt, 
und stellt diese in einen Zusammenhang der aussagt dass es unmöglich ist Energie zu erzeugen oder zu vernichten (\cite[S. 57]{Baehr.1966}).
Für zu Bilanzierende Systeme bedeutet das, dass die Energie in einem abgeschlossenen adiabaten System über die Zeit 
konstant ist (\cite[S. 87]{Rönsch.2015}). 
Adiabat bedeutet in diesem Kontext, dass keine Wärme mit der Umgebung ausgetauscht wird
Für Systeme die in der Lage sind Energie zu speichern, impliziert das nach Rönsch (2015, S. 87f.), 
dass die darin gespeicherte Energie gleich der Differenz aus ein- und austretenden Energieströmen ist. 
Für offene nicht-adiabate Systeme ohne Speicherfähigkeit gilt, dass die differenz der ein- und austretenden Energieströme Null ist (\cite[S. 87f.]{Rönsch.2015}).

Das Verhalten bezüglich Energiespeicherung eines Systems zur Energiebilanzierung wie sie Rönsch (2015) beschreibt lässt sich mathematisch 
mit der Gleichung \eqref{energiebilanzierungsgleichung_Rönsch} darstellen:

\begin{equation}
E_{\text{gespeichert}} = \sum E_{\text{eingang}} - \sum E_{\text{ausgang}}
\label{energiebilanzierungsgleichung_Rönsch}
\end{equation}

\begin{description}
    \item \(E_{\text{gespeichert}}\): Im System gespeicherte Energie.
    \item \(E_{\text{eingang}}\): Energie eines eintretenden Energiestroms.
    \item \(E_{\text{ausgang}}\): Energie eines austretenden Energiestroms.
    Für offene nicht-adiabate Systeme ohne Energiespeicher gilt:
    \[
    E_{\text{gespeichert}} = 0
    \]
    In diesem Fall ist die zugeführte Energie gleich der verbrauchten Energie:
    \[
    \sum E_{\text{eingang}} = \sum E_{\text{ausgang}}
    \]
\end{description}

Diese Gleichung beschreibt einen allgemeinen Ansatz zur Energiebilanzierung der gesamten gespeicherten Energie eines Systems. 
Im rahmen der Bilanzierung von Systemen kann jedoch die Notwendigkeit einer detaillierteren Bilanzierung von einzelnen Zustandsgrößen 
im System notwendig sein. 
Nach den Aussagen Ahrendts (2014, Kapitel 1.5) lässt sich für jede Mengenartige Zustandsgröße die über die 
Grenze eines Systems transportiert wird eine Bilanz aufstellen. 
Diese Bilanz ist abhängig von ein- und austretenden Strömen und den im System enthaltenen Energiequellen und -senken und ermittel die 
Geschwindigkeit der Änderung des Bestands der zu Bilanzierenden Zustandsgröße im System (\cite[Kapitel 1.5]{Ahrendts.2014}).

Ahrendts (2014, Kapitel 1.5) beschreibt die Bilanzierungsgleichung einer Zustandsgröße \eqref{BilanzierungsgleichungAhrendt} wiefolgt:

\begin{equation}
    dX_{\text{j}}/d\tau = (\sum \dot{X}_{\text{j,e}} - \sum \dot{X}_{\text{j,a}}) + (\dot{X}_{\text{j,Quell}} - \dot{X}_{\text{j,Senk}})
    \label{BilanzierungsgleichungAhrendt}
\end{equation}

\begin{description}
    \item \(X_{\text{j}}\): Zustandsgröße.
    \item \(X_{\text{j,e}}\): Über Systemgrenze zufließende Zustandsgröße.
    \item \(X_{\text{j,a}}\): Über Systemgrenze abfließende Zustandsgröße.
    \item \(X_{\text{j,Quell}}\): Quelle der Zustandsgröße im System.
    \item \(X_{\text{j,Senk}}\): Senke der Zustandsgröße im System.
\end{description}

mit der definition des Stroms der Zustandsgröße \(X_{\text{j}}\):

\begin{equation}
    \dot{X}_{\text{j}} = \lim_{\Delta\tau \to 0} \Delta X_{\text{j}}/ \Delta\tau
    \label{BilanzierungsgleichungAhrendtStrom}
\end{equation}

\begin{description}
    \item \(X_{\text{j}}\): Zustandsgröße.
    \item \(\Delta X_{\text{j}}\): Menge der Größe \(X_{\text{j}}\) im Zeitintervall \(\Delta \tau\).
    \item \(\Delta \tau\): Zeitintervall.
\end{description}

Die Gleichungen \eqref{BilanzierungsgleichungAhrendt} und \eqref{BilanzierungsgleichungAhrendtStrom} beschreibt die Geschwindigkeit der Änderung des 
Bestands der Größe \(X_{\text{j}}\) als Summe der Differenzen zwischen den über die Systemgrenz zu- und abfließenden Ströme der Zustandsgröße 
\(X_{\text{j}}\) und den Quell- und Senkenströmen der Größe \(X_{\text{j}}\) innerhalb des Systems. 

% Detaillierter Beschreiben und Schlussfolgerungen ziehen

% Wie kann Zeitliche Aggregation in definition Integriert werden?

\subsubsection{Energieströme}
% Wie realisiert man Ströme (Zeitreihe->Strom)
% Praxisbeispiele
% (praktische Herausforderungen)

\subsubsection{Energiequellen und -senken}
% Was charakterisiert Energiequellen und -senken
% Praxisbeispiele
% (praktische Herausforderungen)

% \subsubsection{Definition der Energiewirtschaft}
