\subsection{Bilanzierung}
Die Energiebilanz nach Rönsch (2015) lässt sich mathematisch mit der folgenden Gleichung darstellen:

\[
E_{\text{gespeichert}} = \sum E_{\text{quelle}} - \sum E_{\text{senke}}
\]

\textbf{Erläuterung der Gleichung:}
\begin{itemize}
    \item \(E_{\text{gespeichert}}\): Energie, die im System gespeichert wird.
    \item \(\sum E_{\text{quelle}}\): Summierte zugeführten Energie, der Energiequellen.
    \item \(\sum E_{\text{senke}}\): Summe aller Energiesenken, also der verbrauchten Energie.
\end{itemize}

Für abgeschlossene Systeme ohne Energiespeicher gilt:
\[
E_{\text{gespeichert}} = 0
\]
In diesem Fall ist die zugeführte Energie gleich der verbrauchten Energie:
\[
\sum E_{\text{quelle}} = \sum E_{\text{senke}}
\]