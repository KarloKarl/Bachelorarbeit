\subsection{Methodik zur Erfüllung von ISO 50001 Anforderungen}

\subsubsection{Integration von relevanten Variablen}
Relevante Variablen werden von der DIN EN ISO 50001:2018-12 (Kapitel 3.4.9) als quantifizierbarer Faktor, der die energiebezogene Leistung wesentlich beeinflusst sich 
routinemäßig ändert definiert. 
Die relevanten Variablen dürfen gemäß E DIN ISO 50006:2024-07 entweder direkt gemessen oder aus Messungen abgeleitet werden (\cite[S. 18]{DIN50006.2024}).
Beispiele für relevante Variablen sind Wetterbedingungen, Betriebsbedingungen wie Innenraumtemperatur oder Lichtstärke und Arbeitsstunden (\cite[Kapitel 3.4.9]{DIN50001.2018}).
Die Integration von relevanten ist einer der Grundlegenden Anforderungen zur erfüllung von Anforderungen der DIN EN ISO 50001:2018-12.
So sollen im rahmen der Bestimmtung der EnPI-Grenzen Organisationen die für die einzelnen Grenzen relevanten Variablen bestimmen (\cite[S. 17]{DIN50006.2024}).
Für die statistische Analyse der EnPI-Werte ist es notwendig dass der Energieverbrauch und die Daten der zugehörigen relevanten Variablen 
die gleichen Zeitintervalle umfassen (\cite[S. 20]{DIN50006.2024}).
Zur ermittlung wesentlicher Energieeinsätze ist ebenfalls nicht nur die Messung deren Energieverbrauchs notwendig, sondern auch die Messung und Überwachung der 
relevanten Variablen bezüglich SEUs (\cite[S. 23]{DIN50001.2018}). 
Die erfassung dient unter anderem der Bewertung der energiebezogenen Leistung unter gleichwertigen Bedingungen (\cite[S. 8]{DIN50006.2024}). 
Die Auswirkungen der relevanten Variablen werden unter Anwendung des Prozesses der Normalisierung berücksichtigt (\cite[S. 8]{DIN50006.2024}).
Die identifikation relevanter Variablen kann unter der Erwägung von Eingangsgrößen, Prozessen, Ergebnisgrößen und der Umgebung angegangen werden 
(vgl. Abbildung \eqref{fig:Erwägungen_Identifizierung_Rel_Var}).

\begin{figure}[H]
    \centering
    \includegraphics[width=1\textwidth]{../../Ressourcen/Abbildungen/Erwägungen_relevanter_Variablen.jpg}
    \caption{Erwägungen für die Identifizierung von Variablen. (Dargestellt von der E DIN ISO 50006:2024-07 (S. 18))}
    \label{fig:Erwägungen_Identifizierung_Rel_Var}
\end{figure}



\subsubsection{Datengetriebener Ansatz}

Die DIN EN ISO 50001:2018-12 schreibt einen Datengetriebenen Ansatz vor indem Sie von Organisationen welche die Norm umsetzen möchten fordert dass ein Plan zur 
Überwachung und Messung der Hauptmerkmale erstellt und dokumentiert wird (\cite[S. 30ff.]{DIN50001.2018}).
Um die genauen Vorgaben der DIN EN ISO 50001:2018-12 zur ermittlung von Energieeinsätzen zu formulieren, wurde die E DIN ISO 50006:2024-07 veröffentlicht,
welche sich mit der Messung der energiebezogenen Leistung im Rahmen der DIN EN ISO 50001:2018-12 befasst (\cite[S. 1]{DIN50006.2024}).

Organisationen sollen nach E DIN ISO 50006:2024-07 (Kapitel 5.1) Arten des Energieeinsatzes identifizieren und zum einen deren aktuellen, sowie früheren 
Energieverbrauch, zum anderen die aktuelle und frühere Energieeffizienz auf Basis von Messungen und anderen Daten bewerten. 
SEUs werden dann anhand der Analyse dieser Informationen, unter berücksichtigung von Faktoren die die energiebezogene Leistung beeinflussen, 
identifiziert (\cite[Kapitel 5.1]{DIN50006.2024}). 
Die Komplexität der Umsetzung ist dabei nicht vorgeschrieben und kann von einfachen Zählwerten bis hin zu umfangreichen Werten aus Überwachungs- und Messsystemen mit
Softwareanwendung reichen (\cite[S. 36]{DIN50001.2018}).
Folglich bestimmt die Komplexität der Energiedatensammlung auch die potentielle Komplexität der Abbildung und Energiebilanzierung 
des Organisationskontext über Bilanzräume. 
Die Datengetriebenen Ermittlung von Energieeinsätzen in Bilanzräumen fordert folglich die Energiedatensammlung der Energieeinsätze, 
zu der auch die DIN EN ISO 50001:2018-12 Vorgaben macht.

Bei der Auswahl von EnPIs sollen Organisationen ihre vorhandenen Fähigkeiten zur Messung und Überwachung in Bezug auf den 
Energiverbrauch und relevante Variablen berücksichtigen (\cite[S. 21]{DIN50006.2024}).
So muss eine Organisation die Energieflüsse die eine EnPI-Grenze überschreiten messen, wobei die gemessenen Daten sowohl die 
zugelieferte als auch vor Ort erzeugte Energie berücksichtigt die die EnPI-Grenze überschreitet und gespeichert wird (\cite[S. 17]{DIN50006.2024}).
Folglich beeinflusst die Komplexität der Energiedatensammlung die Menge der abbildbaren Energieleistungskennzahlen. 
Eine Organisation soll also die sich auf den Energieverbrauch beziehenden Daten und relevanten Variablen für jede EnPI spezifizieren und erfassen (\cite[S. 18]{DIN50006.2024}).
Falls einige EnPIs aufgrund begrenzter Daten oder anderen Hürden nicht messbar sein, soll die Organisation die EnPIs bewerten und in Folge überarbeiten oder 
zusätzliche Zähler, Messungen oder Modellierungsverfahren einführen (\cite[S. 18]{DIN50006.2024}).

Die DIN EN ISO 50001:2018-12 (2018, Kapitel 6.6, A.6.6) stellt Qualitätskriterien an die Energiedatensammlung in Organisationen.
Die Norm verpflichtet Organisationen dazu, Hauptmerkmale ihrer Tätigkeiten, die sich auf die energiebezogene Leistung auswirken, zu identifizieren und diese in geplanten
Zeitabständen zu messen, zu überwachen und zu analysieren (\cite[S. 23]{DIN50001.2018}).
So muss eine geeignete Abtastzeit der Datensammlung gewählt werden (\cite[S. 20]{DIN50006.2024}), und im Rahmen von Analysen müssen Einschränkungen der Daten 
wie Genauigkeit, Präzision und Konsistenz der Energiedatenerfassung Rechnung getragen werden (\cite[S. 37]{DIN50001.2018}).  
Da sich die DIN EN ISO 50001:2018-12 auf die Veränderung der energiebezogenen Leistung bezieht ist die Wiederholbarkeit ein wichtigeres Qualitätskriterium der 
Energiedatensammlung als die Präzision der Messung (\cite[S. 3]{Szajdzicki.2017}).