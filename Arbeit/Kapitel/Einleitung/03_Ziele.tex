\section{Ziel der Arbeit}

Das Ziel dieser Arbeit ist die Konzeption, Implementation und Evaluation eines Prototyps, der durch strukturelle Anpassungen und 
Erweiterungen des EMS-EDM Prophet® die Abbildung frei definierbarer Bilanzräume ermöglicht. Der Prototyp soll einen Mehrwert zur 
Erfüllung der Anforderungen der DIN EN ISO 50001:2018-12 bieten und in Organisationen des tertiären Wirtschaftssektors, die EMS-EDM Prophet® nutzen, 
anwendbar sein. 
Der erarbeitete Prototyp soll auf den theoretischen Grundlagen des Energiemanagements basieren und bewährte Ansätze 
aus dem Forschungsgebiet Energiebilanzierung Datenbankseitig abbilden.
Des weiteren soll der Prototyp einen Mehrwert zur Erfüllung von ISO 50001 Anforderungen hinsichtlich der bestimmung von wesentlichen Energieeinsätzen und 
der Bewertung der energiebezogenen Leistung bieten und den Umfang der im Rahmen des EMS-EDM Prophet® addressierten ISO 50001 Anforderungen erweitern.
Außerdem soll der Prototyp praktische Herausforderungen und Gegebenheiten im Anwendungskontext berücksichtigen.


Zur Evaluation des Prototyps soll die Bilanzraumstruktur der Organisation: Fraunhofer IOSB-AST in Ilmenau erarbeitet und im entworfenen Prototyp abgebildet werden.
Der angewendete Prototyp soll hinsichtlich der Erfüllung von DIN EN ISO 50001:2018-12 Anforderungen, und der korrekten Abbildung der theoretischen Grundlagen 
auf qualitative und quantitative Qualitätskriterien evaluiert werden.
Im Rahmen der evaluierung soll die freie Definierbarkeit im Anwendungsgebiet: Organisationen des tertiären Wirtschaftssektors betrachtet werden. 