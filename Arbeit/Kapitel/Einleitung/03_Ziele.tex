\section{Ziel der Arbeit}

Das Ziel dieser Arbeit ist die Konzeption, Implementation und Evaluation eines Prototyps, der durch strukturelle Anpassungen und 
Erweiterungen des EMS-EDM Prophet® die Abbildung frei definierbarer Bilanzräume ermöglicht. Der Prototyp soll einen Mehrwert zur 
Erfüllung der Anforderungen der DIN EN ISO 50001:2018-12 bieten und in Organisationen, die EMS-EDM Prophet® nutzen, anwendbar sein. Die Erarbeitung des Prototyps soll 
auf den theoretischen Grundlagen des Daten- und Energiemanagements basieren und bewährte Ansätze aus den Bereichen verwenden. Außerdem soll der 
Prototyp praktische Herausforderungen in den potentiellen Anwendungsgebieten berücksichtigen und allgemeine Anforderungen an Organisationen zu dessen 
Umsetzung formulieren.


Zur Evaluation des Prototyps soll die Bilanzraumstruktur der Organisation: Fraunhofer IOSB-AST in Ilmenau im entworfenen Prototyp abgebildet werden.
Der angewendete Prototyp soll im Bezug auf die unterstützung bei der Erfüllung der DIN EN ISO 50001:2018-12 Anforderungen auf qualitative und quantitative 
Qualitätskriterien evaluiert werden.
Außerdem soll die freie Definierbarkeit und die praktische Anwendbarkeit des Prototyps evaluiert werden.
