\section{Aufbau der Arbeit}
Der Aufbau der Arbeit umfasst drei Hauptabschnitte: die theoretischen Grundlagen, die Konzeption und Implementierung eines Lösungsansatzes sowie dessen 
Evaluation.
\begin{enumerate}
    \item \textbf{Theoretische Grundlagen und Stand der Wissenschaft}
    
    Die praxisnahe Problemstellung erfordert eine anwendungsorientierte Forschung. 
    Aufgrund des spezifischen Anwendungskontexts EMS-EDM Prophet® existieren nur wenige vergleichbare Ansätze.
    Daher werden im theoretischen Teil der Arbeit drei Themenbereiche betrachtet: Grundlagen von Bilanzräumen und Bilanzierung, Energiemanagement nach
    DIN EN ISO 50001:2018-12 mit den einhergehenden praktischen Herausforderungen bei der Umsetzung und geeigneten Ansätzen des Datenmanagements zur Abbildung
    von Bilanzräumen in relationalen Datenbanken.
    
    Für die Erarbeitung der theoretischen Grundlagen des Energiemanagements werden in diesem Kapitel der Arbeit die DIN EN ISO 50001:2018-12, damit verbundene Normen und
    Basiswissen aus für den Problemraum relevanter Fachliteratur analysiert. Auf dieser Basis werden theoretische Konzepte und Anforderungen aus dem
    Problemraum abgeleitet, die für die Lösung der Forschungsfrage relevant sind.
    
    Ein zentraler Aspekt der theoretischen Forschung dieser Arbeit ist die Interdisziplinarität. Die Verbindung von Energiemanagement und Datenmanagement
    ist entscheidend, um eine umfassende und praxisnahe Lösung zu entwickeln. Durch die Integration von Erkenntnissen und Methoden aus beiden Bereichen
    wird sichergestellt, dass die theoretischen Grundlagen sowohl die für Bilanzräume relevanten Anforderungen der DIN EN ISO 50001:2018-12 als auch die technischen
    Herausforderungen der Datenbankmodellierung abdecken. Diese interdisziplinäre Herangehensweise ermöglicht es, eine fundierte theoretische Basis zu
    schaffen, die sowohl den praktischen als auch den wissenschaftlichen Anforderungen gerecht wird.
    
    Aufgrund der kleinen Menge an vergleichbaren Ansätzen fließen alternative Lösungsansätze im Rahmen der Erarbeitung der theoretischen Grundlagen nur teilweise ein.
    Der Stand der Wissenschaft wird ausschließlich im Themenbereich Ansätze des Datenmanagements zur Abbildung von Bilanzräumen in relationalen
    Datenbanken behandelt.  

    \item \textbf{Konzeption und Implementation des Prototyps}

    Basierend auf den Forschungsergebnissen des theoretischen Teils der Arbeit wird im zweiten Kapitel der Arbeit eine Lösung für den Problemraum
    konzipiert und implementiert.
    Dabei wird das Konzept begründet und, wenn fertiggestellt, in EMS-EDM Prophet® implementiert.
    Als Grundlage der strukturellen Anpassungen dient der Ausgangszustand der Datenbankstruktur von EMS-EDM Prophet®.
    Dieser wird im Rahmen dieses Kapitels beschrieben und schematisch dargestellt.
    
    Das Konzept basiert auf den theoretischen Grundlagen zu Bilanzräumen, Bilanzierung und Energiemanagement nach DIN EN ISO 50001:2018-12.
    
    Zur Konzipierung der strukturellen Anpassungen und Erweiterungen des Datenbankmodells werden herausgearbeitete bewährte Ansätze zur Datenbankmodellierung von
    Bilanzräumen verwendet.
    
    Außerdem werden Anforderungen an den Organisationskontext zur praktischen Umsetzung des Prototyps auf Grundlage des entworfenen Konzepts und der
    im Theorieteil herausgearbeiteten praktischen Herausforderungen formuliert.

    
    \item \textbf{Evalutation des Prototyps}
    
    Im letzten Hauptkapitel der Arbeit wird der entworfene Prototyp evaluiert.
    Dazu wird die Bilanzraumstruktur des Fraunhofer IOSB-AST in Ilmenau erarbeitet und im entworfenen Prototyp abgebildet.

    Auf Grundlage des angewandten Prototyps sollen nun mehrere Aspekte evaluiert werden.
    Die Evaluation prüft, wie gut der Prototyp die Organisation bei der Erfüllung der DIN EN ISO 50001:2018-12 unterstützt und 
    inwiefern frei definierbare Bilanzräume umsetzbar sind.

    Die Evaluation basiert auf Qualitätskriterien, die im Methodik-Abschnitt aus den theoretischen Grundlagen abgeleitet wurden.
\end{enumerate}