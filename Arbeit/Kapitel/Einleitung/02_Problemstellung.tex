\section{Problemstellung}
\subsection{Problembeschreibung}
\textbf{Forschungsfrage:} "Welche strukturellen Erweiterungen und Anpassungen müssen auf Datenbankebene in EMS-EDM PROPHET® konzipiert und implementiert 
werden, um eine frei definierbare Abbildung von Bilanzräumen zu ermöglichen, die Organisationen bei der Erfüllung der ISO 50001 unterstützt?"

Die breite Anwendbarkeit der in der DIN EN ISO 50001:2018-12 gestellten Anforderungen auf Organisationen führt zu Anforderungen an die Abbildbarkeit von 
frei definierbaren energiebezogenen Organisationsstrukturen im Energiemanagementsystem.

Ein Teilaspekt zur Umsetzung der DIN EN ISO 50001:2018-12 im Rahmen der Energiebilanzierung ist die Abbildung von energiebezogenen Bereichen in 
Organisationen im Energiemanagementsystem und soll im Rahmen dieser Arbeit betrachtet werden.

Die aktuelle Datenbankstruktur von EMS-EDM Prophet® steht vor dem Problem, frei definierbare Bilanzräume abzubilden.
Um dieses Problem zu lösen, sind strukturelle Änderungen und Erweiterungen der Datenbank notwendig.
Somit besteht das zentrale Problem dieser Arbeit darin, ein System zur Abbildung frei definierbarer Bilanzräume auf Datenbankebene unter Berücksichtigung 
der von der DIN EN ISO 50001:2018-12 gestellten Anforderungen an Bilanzräume und damit verbundene Themenkomplexe in EMS-EDM Prophet® zu konzipieren und 
zu implementieren.

Die Problemlösung umfasst alle Aspekte, die auf Grundlage der Vorgaben der Norm sowie praktischer Gegebenheiten konzipiert und auf Datenbankebene 
umgesetzt werden müssen, um EMS-EDM PROPHET® so zu erweitern, dass das System in der Lage ist, Organisationen bei der Erfüllung der ISO 50001 zu 
unterstützen. Dies gilt insbesondere für Anforderungen, die durch die Abbildung von Bilanzräumen adressiert werden können.

Aufgrund der Anwendbarkeit der DIN EN ISO 50001:2018-12 auf alle Organisationen ist die freie Definierbarkeit der Bilanzräume ein Qualitätskriterium des zu 
entwerfenden Systems und spielt bei der Beantwortung der Forschungsfrage eine zentrale Rolle.
Die breite Anwendbarkeit der Norm impliziert außerdem die Notwendigkeit, praktische Herausforderungen beim Einsatz der Lösung zu berücksichtigen und 
Anwendungsgebiete des entworfenen Konzepts zu betrachten, um der praktischen Relevanz dieser Arbeit gerecht zu werden.

\subsection{Praktische Relevanz des Problemraums} 
Das beschriebene Problem weist eine praktische Relevanz auf, da es die Herausforderungen der DIN EN ISO 50001:2018-12
im Energiemanagement von Organisationen adressiert.
Die bestehenden Anforderungen der DIN EN ISO 50001:2018-12 und der aktuelle Zustand von EMS-EDM Prophet® stellen praxisnahe Qualitätskriterien an die Abbildung von Bilanzräumen.
Eine Herausforderungen besteht darin, ein Datenbankmodell zu entwickeln, das diese Anforderungen erfüllt und gleichzeitig praxisnah und umsetzbar ist.
Die Berücksichtigung von aus der Praxis abgeleiteten Anforderungen ist dabei unerlässlich.
Dies verdeutlicht die Notwendigkeit einer Methodik, die sowohl theoretische als auch praktische Aspekte integriert.
Die Integration der Lösung in EMS-EDM Prophet® stellt sicher, dass sie in bestehenden Organisationen nutzbar ist und deren Energiemanagement unterstützt.

\subsection{Wissenschaftliche Relevanz des Problemraums} 
Die Problemstellung weist eine wissenschaftliche Relevanz auf, da im Zuge der Erarbeitung einer Lösung Methoden des Datenmanagements im Kontext der 
Modellierung von Energiebilanzräumen angewandt werden.
Dabei werden die in EMS-EDM Prophet® bestehenden Methoden um neue Ansätze zur Modellierung von Bilanzräumen erweitert.
Diese Erweiterungen tragen zur wissenschaftlichen Diskussion über Datenmanagementstrategien im Energiemanagement bei und bieten neue Perspektiven für die 
Integration von Bilanzräumen in datenbankbasierte Systeme.
Darüber hinaus fördert die Arbeit den interdisziplinären Austausch zwischen den Bereichen Energiemanagement und Datenbankmodellierung, indem sie 
theoretische Konzepte mit praktischen Anwendungen verknüpft.
Die entwickelten methodischen Ansätze und Modelle können als Grundlage für zukünftige wissenschaftliche Untersuchungen dienen und die Weiterentwicklung 
von Energiemanagementsystemen unterstützen.