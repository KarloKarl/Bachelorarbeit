\section{Problemstellung}
\subsection{Problembeschreibung}
\textbf{Forschungsfrage:} "Welche strukturellen Erweiterungen und Anpassungen müssen auf Datenbankebene in EMS-EDM PROPHET® vorgenommen werden, um das 
Energiemanagementsystem zur Abbildung von frei definierbaren Bilanzräumen zu ermächtigen, die Organisationen des tertiären Wirtschaftssektors bei der 
Erfüllung von Anforderungen der ISO 50001 unterstützt?"

Die DIN EN ISO 50001:2018-12 stellt Organisationen vor die Herausforderung, eine fortlaufende Verbesserung ihrer energiebezogenen Leistung nachzuweisen. 
In diesem Kontext spielen die Anforderungen an ein Energiemanagementsystem eine zentrale Rolle. Bilanzräume, die auf dem thermodynamischen Konzept der 
Bilanzierung basieren, bieten Potenzial, Organisationen bei der Erfüllung der Normvorgaben zu unterstützen, indem sie systematisch in das Energiemanagementsystem 
integriert werden.  

Das Energiemanagement EMS-EDM PROPHET® steht vor dem Problem, frei definierbare Bilanzräume abzubilden und somit zur Erfüllung von teilen der Anforderungen 
der DIN EN ISO 50001:2018-12 beizutragen.
Um dieses Problem zu lösen, sind strukturelle Änderungen und Erweiterungen der Datenbank notwendig.
Somit besteht das zentrale Problem dieser Arbeit darin, Anpassungen und Erweiterungen am Datenbanksystem von EMS-EDM Prophet® zur Abbildung 
frei definierbarer Bilanzräume zu konzipieren und zu implementieren um Teile der von der DIN EN ISO 50001:2018-12 gestellten Anforderungen, die im Rahmen dieser 
Arbeit erarbeitet werden zu erfüllen.

Die Problemlösung umfasst alle Aspekte, die auf Grundlage der Vorgaben der Norm sowie praktischer Gegebenheiten konzipiert und auf Datenbankebene 
umgesetzt werden müssen, um EMS-EDM PROPHET® so zu erweitern, dass das System in der Lage ist, Organisationen bei der Erfüllung der ISO 50001 zu 
unterstützen. Dies gilt insbesondere für Anforderungen, die durch die Abbildung von Bilanzräumen adressiert werden können.

Aufgrund der Anwendbarkeit der DIN EN ISO 50001:2018-12 auf alle Organisationen ist die freie Definierbarkeit der Bilanzräume ein Qualitätskriterium des zu 
entwerfenden Systems und spielt bei der Beantwortung der Forschungsfrage eine zentrale Rolle.
Die breite Anwendbarkeit der Norm impliziert außerdem die Notwendigkeit, praktische Herausforderungen beim Einsatz der Lösung zu berücksichtigen und 
Anwendungsgebiete des entworfenen Konzepts zu betrachten, um der praktischen Relevanz dieser Arbeit gerecht zu werden.

Die Menge aller Organisationen für die die DIN EN ISO 50001:2018-12 eine Relevanz hat ist aufgrund ihrer Breiten Anwendbarkeit sehr groß und divers. 
Das wirkt sich auch auf die Anforderungen an die zu entwerfende Problemlösung aus. 
Um den Umfang der Arbeit zu reduzieren und die Präzision und Tiefe der Arbeit zu erhöhen befasst sich diese Forschungsarbeit mit Organisationen des tertiären Wirtschaftssektors.

\subsection{Praktische Relevanz des Problemraums} 
Das beschriebene Problem weist eine praktische Relevanz auf, da es die Herausforderungen der DIN EN ISO 50001:2018-12
im Energiemanagement von Organisationen adressiert.
Die bestehenden Anforderungen der DIN EN ISO 50001:2018-12 und der aktuelle Zustand von EMS-EDM Prophet® stellen praxisnahe Qualitätskriterien an die Abbildung von Bilanzräumen.
Eine Herausforderungen besteht darin, ein Konzept zur Änderung und Erweiterung des bestehenden Datenbankmodell zu entwickeln, das diese Anforderungen erfüllt und gleichzeitig praxisnah und umsetzbar ist.
Die Berücksichtigung von aus der Praxis abgeleiteten Anforderungen ist dabei unerlässlich.
Dies verdeutlicht die Notwendigkeit einer Methodik, die sowohl theoretische als auch praktische Aspekte integriert.
Die Integration der Lösung in EMS-EDM Prophet® stellt sicher, dass sie in bestehenden Organisationen nutzbar ist und deren Energiemanagement unterstützt.

\subsection{Wissenschaftliche Relevanz des Problemraums} 
Die Problemstellung weist eine wissenschaftliche Relevanz auf, da im Zuge der Erarbeitung einer Lösung Methoden des Datenmanagements im Kontext der 
Modellierung von Energiebilanzräumen angewandt werden.
Dabei werden die in EMS-EDM Prophet® bestehenden Methoden um neue Ansätze zur Modellierung von Bilanzräumen erweitert.
Diese Erweiterungen tragen zur wissenschaftlichen Diskussion über Datenmanagementstrategien im Energiemanagement bei und bieten neue Perspektiven für die 
Integration von Bilanzräumen in datenbankbasierte Systeme.
Darüber hinaus fördert die Arbeit den interdisziplinären Austausch zwischen den Bereichen Energiemanagement und Datenbankmodellierung, indem sie 
theoretische Konzepte mit praktischen Anwendungen verknüpft.
Die entwickelten methodischen Ansätze und Modelle können als Grundlage für zukünftige wissenschaftliche Untersuchungen dienen und die Weiterentwicklung 
von Energiemanagementsystemen unterstützen.