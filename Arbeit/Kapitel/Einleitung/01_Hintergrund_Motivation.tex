\section{Hintergrund und Motivation}
Angesichts wachsender Umweltbelastungen und der Notwendigkeit nachhaltiger Praktiken spielt das Energiemanagement eine immer bedeutendere Rolle.
Diese Arbeit untersucht die Entwicklung einer datenbankseitigen Lösung zur Abbildung frei definierbarer Bilanzräume im Energiemanagementsystem
EMS-EDM PROPHET® nach DIN EN ISO 50001:2018-12.
Sie wird durch das Potenzial, die energiebezogene Leistung und Energieeffizienz von Organisationen durch die Erfüllung ausgewählter Kriterien 
der DIN EN ISO 50001:2018-12 zu verbessern, motiviert. 
Bilanzräume stellen das zentrale Konzept der Arbeit dar und werden im Rahmen dieser als Einheiten betrachtet, die zur digitalen Abbildung 
von Organisationsstrukturen im Energiemanagement und als administrative Grenze zur Bilanzierungsrechnung dienen.
Die Adressierung der Arbeit auf die freie definierbare Gestaltung der Bilanzräume soll eine Möglichkeit bieten, der Diversität von Organisationen gerecht zu werden
und einen Einsatz der Forschungsergebnisse in Organisationen mit dem EDM-EMS-Prophet® ermöglichen.
Die Untersuchung soll zur Weiterentwicklung nachhaltiger Energiemanagementpraktiken beitragen und Einblicke in die Integration
technischer Lösungen in bestehende Systeme bieten.

Ein wesentlicher Fokus dieser Arbeit liegt auf der DIN EN ISO 50001:2018-12, einer Norm der Internationalen Organisation für Normung (ISO),
die Anforderungen an Energiemanagementsysteme festlegt. Diese Norm ist universell einsetzbar, unabhängig von Größe, Art oder Standort der Organisation (\cite[S. 10]{DIN50001.2018}),
und dient der fortlaufenden Verbesserung der energiebezogenen Leistung. (\cite[S. 7]{DIN50001.2018}).
Um die Anforderungen der DIN EN ISO 50001:2018-12 zu erfüllen, müssen Organisationen den kontinuierlichen Fortschritt ihrer energiebezogenen Leistung nachweisen, wobei
die Norm keine spezifischen Zielniveaus vorgibt. (\cite[S. 10]{DIN50001.2018}).

Die Umsetzung der DIN EN ISO 50001:2018-12 in Organisationen bringt sowohl operationale als auch organisatorische Herausforderungen mit sich [S. 11](\cite{Marimon.2017}).
Dennoch lag im Jahr 2023 in 24.924 Organisationen weltweit ein Zertifikat nach DIN EN ISO 50001:2018-12 vor (\cite{InternationalOrganizationforStandardization.2023}).
Dies ist bemerkenswert, da die Erfüllung der Normanforderungen voraussichtlich etwa 60 \% des globalen Energieverbrauchs beeinflussen
kann (\cite{InternationalOrganizationforStandardization.2011}, zitiert nach \cite[S. 1]{Marimon.2017}). Darüber hinaus entstehen für Organisationen durch
die Einführung der Norm signifikante Vorteile.

Zum einen können nach Aussagen der DIN EN ISO 50001:2018-12 (2018, S. 9) ökonomische Vorteile wie Energieeinsparungen erzielt werden, wodurch Organisationen einen Wettbewerbsvorteil
aufgrund sinkender Energiekosten erlangen können. Zum anderen ergeben sich operationale Vorteile wie eine gesteigerte Produktivität, verbesserte Qualität
und ein strukturierter Ansatz zur Prozessoptimierung (\cite{Marimon.2017}). Des Weiteren kann die Umsetzung der DIN EN ISO 50001:2018-12 dazu beitragen, die allgemeinen
Klimaschutzziele zu erreichen (\cite{DIN50001.2018}). Dies unterstreicht die gesellschaftliche Bedeutung der Norm, insbesondere angesichts der Herausforderungen
des Klimawandels.

Die Umsetzung der DIN EN ISO 50001:2018-12 basiert auf dem PDCA-Zyklus (Plan, Do, Check, Act), der Organisationen einen strukturierten Rahmen für die fortlaufende
Verbesserung der energiebezogenen Leistung bieten soll (\cite[S. 7f.]{DIN50001.2018}).
Während die Norm in erster Linie Anforderungen auf Managementebene formuliert, verweist sie auch auf technische Normen wie die E DIN ISO 50006:2024-07, die unter anderem
spezifische Anforderungen an Energieleistungskennzahlen und energetische Ausgangsbasen definiert (\cite{DIN50006.2024,DIN50001.2018}).