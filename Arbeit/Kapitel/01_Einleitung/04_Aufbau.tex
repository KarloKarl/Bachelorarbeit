\section{Aufbau der Arbeit}
Diese Arbeit ist so konzipiert, dass Sie die theoretischen Grundlagen des Problemraums erfasst und Nutzen sowie Herausforderungen im Anwendungsgebiet: 
EMS nach DIN EN ISO 50001:2018-12 erarbeitet. 
Basierend auf den theoretischen Grundlagen im Anwendungsbereich und den bestehenden Methoden und Ansätzen des Datenmanagements 
in EMS-EDM Prophet® wird eine Lösung der Forschungsfrage auf Datenbankebene des Energiemanagementsystems konzipiert, 
implementiert und evaluiert.
Der Aufbau der Arbeit umfasst drei Hauptabschnitte: die theoretischen Grundlagen und der Stand der Wissenschaft, die Konzeption und Implementation, und die 
Evaluation.

\begin{enumerate}
    \item \textbf{Theoretische Grundlagen und Stand der Wissenschaft}
    
    Die praxisnahe Problemstellung erfordert eine anwendungsorientierte Forschung unter Berücksichtigung der Interdisziplinarität. 
    Im theoretischen Teil der Arbeit werden zwei Themenbereiche betrachtet: Grundlagen der Energiebilanzierung unter Nutzung von Bilanzräumen und  
    Energiemanagementsysteme nach DIN EN ISO 50001:2018-12. In beiden Tehemenbereichen findet die erarbeitung der Grundlagen unter beachtung des Anwendungsgebiets: 
    Organisation im tertiären Wirtschaftssektor statt. 
    
    Für die Erarbeitung der theoretischen Grundlagen des Energiemanagements werden im ersten Hauptabschnitt der Arbeit die DIN EN ISO 50001:2018-12, 
    damit verbundene Normen und Basiswissen aus für den Problemraum relevanter Fachliteratur analysiert.
    Außerdem werden wissenschaftliche Arbeiten aus verwandten Problemräumen analysiert und in den Kontext dieser Arbeit gesetzt.
    Auf dieser Basis werden theoretische Konzepte und Anforderungen aus dem Problemraum abgeleitet, die für die Lösung der Forschungsfrage relevant sind.
    
    Der erste Hauptabschnitt der Arbeit hat somit eine zentrale Bedeutung zum erreichen der Interdisziplinarität der Forschung.
    Die umfangreiche erarbeitung von Konzepten des Energiemanagements, Anforderungen von Anforderungen der ISO 50001 und den Einsatzmöglichkeiten 
    von Bilanzräumen zur praxisnahen erfüllung dieser Anforderungen auf basis der Konzepte stellen eine detaillierte Analyse des Anwendungsbereichs dieser 
    Forschung dar.
    Diese Detaillierte Analyse, ohne technische Perspektive der Datenbankmodellierung, ist notwendig um der Interdisziplinarität des Problemraums aus Sicht des 
    Energiemanagements und der Energiebilanzierung gerecht zu werden. 
    Die erarbeiteten Grundlagen des Anwendungsbereichs: Energiemanagent wird im nächsten Hauptabschnitt, der Konzeption und Implementation, 
    aufgegriffen und aus einer technischen Sicht des Datenbankmanagements betrachtet.

    \item \textbf{Konzeption und Implementation des Prototyps}

    Basierend auf den Forschungsergebnissen des theoretischen Teils der Arbeit wird im zweiten Kapitel der Arbeit eine Lösung für den Problemraum
    konzipiert und implementiert.
    Um der Interdisziplinarität aus Sicht der Datenbankmodellierung gerecht zu werden, wird der IST-Zustand des EMS-EDM Prophet® analysiert und es werden 
    bereits bestehende Ansätze der Datenbankmodellierung die den Problemraum addressieren aufgezeigt.
    Unter berücksichtigung der aufgezeigten Ansätze wird der Prototyp zur Problemlösung konzipiert und EMS-EDM Prophet® im implementiert.
    Im Zuge dessen ist die konzeption und integration von Ansätzen des Datenbankmanagements, die noch nicht im EMS-EDM Prophet® bestehen 
    notwendig.
    Das Konzept addressiert die im ersten Hauptabschnitt erarbeiteten Erkenntnisse des Anwendungsbereichs: Energiemanagement und Energiebilanzierung 
    und stellt eine Datenbankseitige abbildung der Grundsätze unter den erarbeiteten Anforderungen der ISO 50001 im kontext von Organisationen des 
    tertiären Wirtschaftssektors zur verfügung.
    
    \item \textbf{Evalutation des Prototyps}
    
    Im dritten Hauptabschnitt der Arbeit wird der entworfene Prototyp evaluiert.
    Im Zuge dessen wird die Bilanzraumstruktur des Fraunhofer IOSB-AST in Ilmenau erarbeitet und im entworfenen Prototyp abgebildet.
    Anhand dieses praktischen Beispiels wird getestet wie korrekt die im ersten Hauptabschnitt erläuterten Methoden des Energiemanagements und der Energiebilanzierung 
    umgesetzt wurden, und wie der Prototyp in der Praxis bei der Erfüllung von ISO 50001 Anforderungen Organisationen unterstützen kann. 
    Dieser Abschnitt der Evaluation findet unter Nutzung quantitativer Qualitätskriterien statt.

    Des weiteren wird grundsätzlich betrachtet, wie im Rahmen der Interdisziplinarität die im ersten Hauptabschnitt erarbeiteten Erkenntnisse aus Sicht 
    des Energiemanagements technisch durch den konzipierten und implementierten Prototyp abgebildet wurde.
    Dabei wird auch analysiert ob und wie die im ersten Hauptabschnitt erörterten Anforderungen der ISO 50001 auf Datenbankebene umgesetzt wurden sind.
    Dieser Abschnitt der Evaluation findet unter Nutzung qualitativer Qualitätskriterien statt.

\end{enumerate}