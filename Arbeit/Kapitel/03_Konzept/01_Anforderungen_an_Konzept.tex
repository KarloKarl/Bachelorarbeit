Dieser Abschnitt befasst sich mit den im Kapitel 2 erarbeiteten Erkenntnissen und formuliert deren Grundlage Anforderungen an das Bilanzraumkonzept.
Zur formulierung der Anforderungen werden Herausforderungen welche sich aus den Grundlagen der Energiebilanzierung und aus den betrachteten Konzepten zur Abbildung 
von Energiebilanzen ergeben betrachtet.
Des weiteren werden Anforderungen auf Grundlage der Angaben der DIN EN ISO 50001:2018-12 zu Energiemanagementsystemen in den Aspekten: 
Energieleistungskennzahlen und wesentliche Energieeinsätze erarbeitet.
Die erarbeiteten Anforderungen sind in Tabelle \eqref{tab:funktionale_anforderungen} dargestellt.


\begin{longtable}{| m{0.06\textwidth} | m{0.16\textwidth} | m{0.38\textwidth} | m{0.4\textwidth} |}
    \caption{Funktionale Anforderungen an das Bilanzraumkonzept auf Basis der theoretischen Erkenntnisse.} \\
    \label{tab:funktionale_anforderungen} \\ 
    
    \hline
    \textbf{Index} & \textbf{Aspekt} & \textbf{Anforderung} & \textbf{Begründung} \\
    \hline
    \multicolumn{4}{|c|}{\textbf{Grundlegende strukturelle Anforderungen an das Datenbankschema}}\\
    \hline
    1 
    & Definition von Energieströmen, -quellen und -senken in Bilanzräumen
    & Im Bilanzraumkonzept \textbf{müssen} für jeden Bilanzraum beliebig viele zu- und abfließende Energieströmen, Energiequellen und Energiesenken definiert werden können. 
    & Die mathematische Beschreibung der Bilanzierung nach Ahrendts (vgl. Gleichung \eqref{BilanzierungsgleichungAhrendt}) beschreibt die Veränderung der bilanzierten 
    Zustandsgröße in abhängigkeit von Zustandsströmen, -quellen und -senken. \\
    \hline
    2
    & Definition des Untersuchungsgegenstands 
    & Im Bilanzraumkonzept \textbf{muss} für jeden Bilanzraum ein Untersuchungsgegenstands definiert werden können. 
    & Der Untersuchungsgegenstand beeinflusst die Systemgrenze einer Bilanz (\cite[S. 109]{Miller.2016}). \\
    \hline
    3
    & Zeitliche Abgrenzung 
    & Das Bilanzraumkonzept \textbf{muss} die Möglichkeit bieten, die Bilanzierung in einem Bilanzraum durch ein Zeitintervall zeitlich abzugrenzen. 
    & Die Zustandsgröße der Bilanz ist nach der Mathematischen Beschreibung von Ahrendts (vgl. Gleichung \eqref{BilanzierungsgleichungAhrendt}) abhängig vom Zeitintervall, 
    in dem Energieströme, -quellen und -senken die Bilanz beeinflussen. \\
    \hline
    4
    & Integration von Energiespeichern 
    & Das Bilanzraumkonzept \textbf{muss} die Abbildung und Integration von energiespeichernden Komponenten ermöglichen. 
    Im zuge der Integration von energiespeichernden Komponenten \textbf{müssen} dem Energiespeicher zu- und abfließende Energieströme 
    abbilden. 
    & Energiespeicher wirken sich nach der Mathematischen Beschreibung einer Energiebilanz nach Rönsch (vgl. Gleichung \eqref{energiebilanzierungsgleichung_Rönsch}) 
    auf das Verhalten der bilanzierten Zustandsgröße aus.
    Energiespeicher nehmen Energie auf und geben sie zeitversetzt ab (\cite[S. 1]{Rathgeber.2018}). \\
    \hline
    5
    & Bewertungs-einheiten eines Bilanzraums 
    & Das Bilanzraumkonzept \textbf{muss} alle Energieströme, -quellen und -senken in der Einheit kWh und derem vielfachen unabhängig von Energieformen und Energieträger 
    in einem Bilanzraum abbilden können. 
    & Die bevorzugte Bewertungseinheit für Energieformen ist kWh und deren Vielfaches (\cite[S. 65]{Konstantin.2023}).\\
    \hline
    6
    & Quantifizierung von Energiedienstleistungen & 
    Das Bilanzraumkonzept \textbf{muss} in Anlehnung an das von Miller (2016) entworfene Konzept der Bewertungsräume aus einem Bilanzraum austretende Energieströme 
    in Form von Energiedienstleistungen mit einer Bewertungseinheit konkretisieren und quantifizieren können. 
    & Das in der Energiewertschöpfungskette (vgl. Abbildung \eqref{fig:Energieflussschema_Posch}) visualisierte Problem:
    dass Energie in unterschiedlichen Formen auftritt und insbesondere im Rahmen der Bilanzierung von Gebäudeenergie zufließende Energieströme in Form von messbarer 
    Endenergie auftreten während die abfließenden Energieströme in Form nicht messbarer Energiedienstleistungen auftreten können wird durch die von 
    Miller (2016) beschriebene Quantifizierung von Energie in Form von Energiedienstleistungen adressiert. \\
    \hline
    7
    & Disaggregation von Bilanzräumen
    & Das Bilanzraumkonzept \textbf{muss} die Disaggregation von Bilanzräumen in Teilbilanzräume abbilden können. 
    & Nach Engelmann (2015) und Miller (2016) haben Bilanzräume die Eigenschaft der Zerlegbarkeit.\\
    \hline
    8
    & Aggregation zufließender Energieströme 
    & Das Bilanzraumkonzept \textbf{soll} in Anlehnung an das von Miller (2016) entworfene Konzept der Bewertungsräume eine Aggregation mehrerer zufließender 
    Energieströme zu einem zufließenden Energiestrom in Bilanzräumen abbilden können. 
    & Mehrere einfließende Energieströme und Energiequellen können zur Deckung des Nutzenergiebedarfs der gleichen Energiedienstleistung beitragen,
    weshalb ihre Aggregation notwendig ist um Rückschlüsse auf den Energieverbrauch einer Energiedienstleistung zu ziehen. \\
    \hline
    9
    & Definierbarkeit der Bilanzraumgrenzen nach Grenzniveaus der E DIN ISO 50006:2024-07 
    & Ein Bilanzraum im Bilanzraumkonzept \textbf{muss} nach allen von der E DIN ISO 50006:2024-07 definierten EnPI-Grenzniveaus (vgl. Abbildung \eqref{fig:EnPI_Grenzniveaus}) abgegrenzt werden können. 
    & Die räumliche und zeitliche Abgrenzung eines Bilanzraums beeinflusst die Berechnung und Vergleichbarkeit von EnPI-Werten (\cite[S. 6]{Hohnhold.2013}). \\
    \hline
    \multicolumn{4}{|c|}{\textbf{Strukturelle Anforderungen an das Datenbankschema}}\\
    \multicolumn{4}{|c|}{\textbf{zur Unterstützung bei DIN EN ISO 50001:2018-12 Anforderungen }}\\
    \hline
    10
    & Definierbarkeit von Energieleistungskennzahlen 
    & Das Bilanzraumkonzept \textbf{muss} die Definition von Energieleistungskennzahlen (EnPIs) in einem Bilanzraum ermöglichen. 
    & Energieleistungskennzahlen sind zentral für die Bewertung der energiebezogenen Leistung im Energiemanagement nach ISO 50001. \\
    \hline
    11
    & Definierbare Arten von Energieleistungskennzahlen
    & Das Bilanzraumkonzept \textbf{muss} in einem Bilanzraum EnPIs der Art: absolute Zahlen sowie Index-, Beziehungs- und Gliederungszahlen definieren können. &
    Unterschiedliche Arten von Kennzahlen bieten unterschiedliche Perspektiven auf die Bewertung eines Bilanzraums. \\
    \hline
    12
    & Integration der energetischen Ausgangsbasis 
    & Das Bilanzraumkonzept \textbf{muss} in einem Bilanzraum für alle definierten EnPI eine energetische Ausgangsbasis bereitstellen und im Bezugszeitraum aggregieren können.  
    & Die Differenz zwischen der Ausgangsbasis und dem EnPI-Wert bestimmt die Verbesserung der Energieeffizienz (vgl. Abbildung \eqref{fig:Beziehung_EnPI_EnB_ISO_50006}).
    Da die Berechnung einer EnB mit einem anhand eines Energiemodells berechnet wird ist die Berechnung der EnBs kein Aspekt dieser Anforderung \\
    \hline
    13
    & Definition von operativen Energiezielen 
    & Das Bilanzraumkonzept \textbf{muss} die Möglichkeit bieten, für einen definierten EnPI ein operatives Energieziel anzugeben und dessen Erfüllung zu evaluieren. 
    & Die Bewertung der Zielerreichung ist essenziell für die kontinuierliche Verbesserung der energiebezogenen Leistung (vgl. Abbildung \eqref{fig:Beziehung_EnPI_EnB_ISO_50006}). \\
    \hline
    14
    & Integration von relevanten Variablen
    & Das Bilanzraumkonzept \textbf{muss} für jeden Bilanzraum die Möglichkeit bieten, relevante Variablen zu definieren. 
    & Die Integration von relevanten Variablen ist einer der Grundlegenden Anforderungen zur erfüllung von Anforderungen der DIN EN ISO 50001:2018-12.
    Sie müssen unter anderem im Rahmen der Definition von EnPI-Grenzen bestimmt werden (\cite[S. 17]{DIN50006.2024})  \\
    \hline
    \multicolumn{4}{|c|}{\textbf{Anforderungen an die Datenauswertung der Bilanzräume}}\\
    \multicolumn{4}{|c|}{\textbf{zur Unterstützung bei DIN EN ISO 50001:2018-12 Anforderungen }}\\
    \hline
    15
    & Zeit-Werte-Verlauf von EnPI und EnB 
    & Das Bilanzraumkonzept \textbf{muss} den Zeit-Werte-Verlauf für alle im Bilanzraum definierten EnPIs und die dazugehörige EnBs ausgeben können. 
    & Eine zeitabhängige Analyse ermöglicht Erkenntnisse über das zeitabhängige Verhalten und den Einfluss relevanter Variablen (\cite[S. 14]{DIN50006.2024}). \\
    \hline
    16
    & Abbildbare Kennzahlenarten 
    & Das Bilanzraumkonzept \textbf{muss} in einem Bilanzraum EnPIs der Art: absolute Zahlen sowie Index-, Beziehungs- und Gliederungszahlen berechnen können. &
    Unterschiedliche Arten von Kennzahlen bieten unterschiedliche Perspektiven auf die Bewertung eines Bilanzraums. \\
    \hline
    17
    & Zeitlichen Aggregation von EnPI-Werten 
    & Für jeden in einem Bilanzraum definierten EnPI \textbf{müssen} die dazugehörigen EnPI-Werten zu einem EnPI-Wert über das im Bilanzraum definierte Zeitintervall aggregieren können. 
    & Die zeitliche Aggregation erlaubt eine Bewertung der energiebezogenen Leistung über einen Berichtszeitraum hinweg und ermöglicht den Vergleich mit dem EnB zur ermittlung der Energiebezogenen Leistung 
    (vgl. Abbildung \eqref{fig:Beziehung_EnPI_EnB_ISO_50006}). \\
    \hline
    18
    & Bestimmung des Energieeinsatzes
    & Für jeden Bilanzraum im Bilanzraumkonzept \textbf{muss} der Energieeinsatz in einem Bilanzzeitraum berechnet werden können. 
    & Auf Grundlage des Vergleichs von Energieeinsätzen können nach DIN EN ISO 50001:2018-12 wesentliche Energieeinsätze auf Grundlage 
    der von der Organisation festgelegten Kriterien identifiziert werden. \\
    \hline
    19
    & Bestimmung des spezifischen Energieeinsatzes
    & Für jeden Bilanzraum im Bilanzraumkonzept der den Energievebrauch einer Energiedienstleistungen bilanziert \textbf{muss} der spezifische Energieeinsatz 
    (vgl. Gleichung \eqref{SepzifischerEnergieinsatzMiller}) in einem Bilanzzeitraum berechnet werden können. 
    & Der spezifische Energieeinsatz kann als Maß der Energieeffizienz bei der Erfüllung einer Energiedienstleistung genutzt werden und somit Bilanzräume mit Potential zur 
    Energieeinsparung aufzeigen. \\
    \hline
    20
    & Auswertung von Energieeinsätzen in Disaggregierten (Teil-)Bilanzrräumen
    & Das Bilanzraumkonzept \textbf{muss} Funktionen zur Analyse des Energieeinsatzes eines Bilanzraums im Verhältnis zu seinen disaggregierten Teilbilanzräumen bereitstellen.
    & Die Analyse des Energieeinsatzes in Teilbilanzräumen im Verhältnis zum Gesamtbilanzraum ermöglicht die Erfassung des Anteils des Energieverbrauchs des 
    Teilbilanzraums am Gesamtenergieverbrauch \\
    \hline 
\end{longtable}
Quelle: Eigene Darstellung basierend auf den Erkenntnisse aus Kaptiel 2: Stand der Forschung und Theoretische Grundlagen.