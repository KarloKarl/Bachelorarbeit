Dieser Abschnitt befasst sich mit den im Kapitel 2 erarbeiteten Erkenntnissen und formuliert deren Grundlage Anforderungen an das zu entwerfende Bilanzraumkonzept.
Zur formulierung der Anforderungen werden Herausforderungen welche sich aus den Grundlagen der Energiebilanzierung und aus den betrachteten Konzepten zur Abbildung 
von Energiebilanzen ergeben betrachtet.
Des weiteren werden Anforderungen auf Grundlage der Angaben der DIN EN ISO 50001:2018-12 zu Energiemanagementsystemen in den Aspekten: 
Energieleistungskennzahlen und wesentliche Energieeinsätze erarbeitet.

\subsection{Energiebilanzierung}
Die mathematische Beschreibung der Bilanzierung einer Zustandsgröße in einem Thermodynamischen System (vgl. Gleichung \eqref{BilanzierungsgleichungAhrendt}) gibt auskunft 
über die Bestandteile die eine Bilanz und somit auch ein Bilanzraum umfassen muss. 
Folglich \textbf{muss} das Bilanzraumkonzept zu- und abfließende Energiestöme, Energiequellen und Energiesenken umfassen.
Da in Gleichung \eqref{BilanzierungsgleichungAhrendt} beliebig viele zu- und abfließende Energieströme, Energiequellen und -senken bilanziert werden können 
\textbf{muss} das Bilanzraumkonzept beliebig viele zu- und abfließende Energieströme, Energiequellen und -senken im Rahmen der Bilanzierung umfassen können.
Aufgrund der Abhängigkeit der Größe der bilanzierten Zustandsgröße (vgl. Gleichung \eqref{BilanzierungsgleichungAhrendt} und \eqref{BilanzierungsgleichungAhrendtStrom}) 
vom Zeitintervall indem Energieströme, -quellen und -senken die Zustandsgröße beeinflussen \textbf{muss} das Bilanzraumkonzept die Möglichkeit bieten die Bilanzierung 
zeitlich abzugrenzen.


Der auf Basis von Rönsch (2015) mathematisch beschriebene Ansatz einer Energiebilanz (vgl. Gleichung \eqref{energiebilanzierungsgleichung_Rönsch}) erweitert die Anforderungen 
da er sich mit dem Verhalten von Systemen mit energiespeicherfähigkeit befasst.
Somit \textbf{muss} das Bilanzraumkonzept die Abbildung und Integration von energiespeichernden Komponenten ermöglichen.
Da ein Energiespeicher nach Rathgeber (2018, S.1) Energie aufnehmen und zu einem späteren Zeitpunkt wieder abgeben kann, \textbf{muss} das Bilanzraumkonzept für jeden 
im Bilanzraum enthaltenen Energiespeicher die dem Energiespeicher zu- und abfließenden Energieströme abbilden.   

Aufgrund der bevorzugten Verwendung der Bewertungseinheit: kWh und deren Vielfachen für alle Energieformen \textbf{muss} das Bilanzraumkonzept dazu fähig sein 
durch zu- und abfließende Energieströme, Energiequellen und -senken entsehende Energiemengen mit der Einheit kWh und derem Vielfachen im Rahmen der Bilanzierung zu 
verarbeiten.

Die Zerlegbarkeit von Bilanzräumen wie Sie von Engerlmann (2015) und Miller (2016) beschrieben wurde impliziert Anforderungen an Beziehungen zwischen Bilanzräumen.
Folglich \textbf{muss} das Bilanzraumkonzept die Hierarchische Disaggregation von eines Bilanzraums abbilden können.
Dabei \textbf{muss} das Bilanzraumkonzept sowohl den gesamten Untersuchungsgegenstand des Bilanzraums abbilden und nachvollziehen können und 
über die Hierarchische Disaggregation die Hierarchie des Untersuchungsgegenstands und die dazugehörigen Energiebilanzen abbilden können. 
Der von Hohnhold und Kai (2013, S. 14) begründete Bedarf der Unterscheidung nach Verbrauchsarten impliziert dass das Bilanzraumkonzept auch 
nach Verbrauchsarten, beziehungsweise Nutzengrößen disaggregiert werden können \textbf{soll}.

Das von Miller (2016) entworfene Konzept der Bilanzräume bietet mit seinem Ansatz der teilung in Aufwands- und Nutzenseite einen Ansatz zur Konzeption eines Bilanzraums 
zur betrachtung der Energieeffizienz.
Der Ansatz bietet unter anderem einen Mehrwert da er die durch die Energiewertschöpfungskette (vgl. Abbildung \eqref{fig:Energieflussschema_Posch}) entstehende Problem:
dass Energie in unterschiedlichen Formen auftritt und insbesondere im Rahmen der Bilanzierung von Gebäudeenergie zufließende Energieströme in Form von Endenergie auftreten 
während die abfließenden Energieströme in Form Energiedienstleistungen auftreten.
In Anlehnung an Miller (2016) \textbf{soll} das Bilanzraumkonzept die Möglichkeit bieten Energiedienstleistungen, die aus dem Bilanzraum abfließende Energieströme sind, 
mit Bewertungseinheiten zu konkretisieren und mit Werten der Bewertungseinheit zu quantifizieren. 
Außerdem \textbf{soll} das Bilanzraumkonzept in Anlehnung an Miller (2016) eine Aggregation mehrerer zufließender Energieströme zur erfüllung der selben Energiedienstleistungen 
abbilden können.

\subsection{ISO 50001}

Die Bewertung der energiebezogenen Leistung durch Energieleistungskennzahlen und damit verbundenen Größen ist eine der betrachteten Aspekte der DIN EN ISO 50001:2018-12.
Da das Bilanzraumkonzept darauf abzielt bei der Erfüllung unter anderem dieses Aspekts der Norm beizutragen ergeben sich Anforderungen für das Konzept auf Grundlage der 
Vorgaben der Norm.
Um alle Arten von Energieleistungskennzahlen (vgl. Abbildung \eqref{fig:Arten_Kennzahlen}) berechnen zu können \textbf{muss} das Bilanzraumkonzept es ermöglichen für 
einen Bilanzraum Absolute Zahlen und Verhältniszahlen des Typs Index-, Beziehungs-, und Gliederungszahlen abzubilden und zu berechnen.

% TODO: Grenzniveaus  EnPIs