Dieser Abschnitt befasst sich mit den im Kapitel 2 erarbeiteten Erkenntnissen und formuliert deren grundlegende Anforderungen an das Bilanzraumkonzept.  
Zur Formulierung der Anforderungen werden Herausforderungen betrachtet, die sich aus den Grundlagen der Energiebilanzierung sowie aus den betrachteten Konzepten zur Abbildung von Energiebilanzen ergeben.  
Des Weiteren werden Anforderungen auf Grundlage der Angaben der DIN EN ISO 50001:2018-12 zu Energiemanagementsystemen in den Aspekten  
Energieleistungskennzahlen und wesentliche Energieeinsätze erarbeitet.  
Die erarbeiteten Anforderungen sind in Tabelle \eqref{tab:funktionale_anforderungen} dargestellt.

\begin{longtable}{| m{0.06\textwidth} | m{0.16\textwidth} | m{0.38\textwidth} | m{0.4\textwidth} |}
    \caption{Funktionale Anforderungen an das Bilanzraumkonzept auf Basis der theoretischen Erkenntnisse.} \\
    \label{tab:funktionale_anforderungen} \\ 
    
    \hline
    \textbf{Index} & \textbf{Aspekt} & \textbf{Anforderung} & \textbf{Begründung} \\
    \hline
    \multicolumn{4}{|c|}{\textbf{Grundlegende strukturelle Anforderungen an das Datenbankschema}}\\
    \hline
    1 
    & Definition von Energieströmen, -quellen und -senken in Bilanzräumen
    & Im Bilanzraumkonzept \textbf{müssen} für jeden Bilanzraum beliebig viele zu- und abfließende Energieströme, Energiequellen und Energiesenken definiert werden können. 
    & Die mathematische Beschreibung der Bilanzierung nach Ahrendt (vgl. Gleichung \eqref{BilanzierungsgleichungAhrendt}) beschreibt die Veränderung der bilanzierten 
    Zustandsgröße in Abhängigkeit von Zustandsströmen, -quellen und -senken. \\
    \hline
    2
    & Definition des Untersuchungsgegenstands 
    & Im Bilanzraumkonzept \textbf{muss} für jeden Bilanzraum ein Untersuchungsgegenstand definiert werden können. 
    & Der Untersuchungsgegenstand beeinflusst die Systemgrenze einer Bilanz (\cite[S. 109]{Miller.2016}). \\
    \hline
    3
    & Zeitliche Abgrenzung 
    & Das Bilanzraumkonzept \textbf{muss} die Möglichkeit bieten, die Bilanzierung in einem Bilanzraum durch ein Zeitintervall zeitlich abzugrenzen. 
    & Die Zustandsgröße der Bilanz ist nach der mathematischen Beschreibung von Ahrendt (vgl. Gleichung \eqref{BilanzierungsgleichungAhrendt}) abhängig vom Zeitintervall, 
    in dem Energieströme, -quellen und -senken die Bilanz beeinflussen. \\
    \hline
    4
    & Integration von Energiespeichern 
    & Das Bilanzraumkonzept \textbf{muss} die Abbildung und Integration von energiespeichernden Komponenten ermöglichen. 
    Im Zuge der Integration von energiespeichernden Komponenten \textbf{müssen} dem Energiespeicher zu- und abfließende Energieströme 
    zugeordnet werden können. 
    & Energiespeicher wirken sich nach der mathematischen Beschreibung einer Energiebilanz nach Rönsch (vgl. Gleichung \eqref{energiebilanzierungsgleichung_Rönsch}) 
    auf das Verhalten der bilanzierten Zustandsgröße aus.
    Energiespeicher nehmen Energie auf und geben sie zeitversetzt ab (\cite[S. 1]{Rathgeber.2018}). \\
    \hline
    5
    & Bewertungseinheiten eines Bilanzraums 
    & Das Bilanzraumkonzept \textbf{muss} alle Energieströme, -quellen und -senken in der Einheit kWh und deren Vielfachen unabhängig von Energieformen und Energieträgern 
    in einem Bilanzraum abbilden können. 
    & Die bevorzugte Bewertungseinheit für Energieformen ist kWh und deren Vielfaches (\cite[S. 65]{Konstantin.2023}).\\
    \hline
    6
    & Quantifizierung von Energiedienstleistungen & 
    Das Bilanzraumkonzept \textbf{muss} in Anlehnung an das von Miller (2016) entworfene Konzept der Bewertungsräume aus einem Bilanzraum austretende Energieströme 
    in Form von Energiedienstleistungen mit einer Bewertungseinheit konkretisieren und quantifizieren können. 
    & In der Energiewertschöpfungskette (vgl. Abbildung \eqref{fig:Energieflussschema_Posch}) tritt Energie in unterschiedlichen Formen auf. Insbesondere im Rahmen der Bilanzierung von Gebäudeenergie 
    fließen Energieströme als messbare Endenergie zu, während die abfließenden Energieströme als nicht messbare Energiedienstleistungen auftreten können. 
    Dieses Problem wird durch die von Miller (2016) beschriebene Quantifizierung von Energie in Form von Energiedienstleistungen adressiert. \\
    \hline
\end{longtable}
Quelle: Eigene Darstellung basierend auf den Erkenntnissen aus Kapitel 2: Stand der Forschung und theoretische Grundlagen.