Im Folgenden wird der aktuelle Zustand des für den Problemraum relevanten Segments des Datenbankschemas des Energiemanagementsystems EDM-EMS-Prophet® in der Version 14.1 erläutert.
Das System basiert auf der \textit{Oracle Database 19c Standard Edition 2} (Release 19.0.0.0.0 – Production). 
Die Struktur des Datenbankschemas ist in Abbildung \eqref{fig:Datenbankschema_Prophet_ZRM} dargestellt.

\subsubsection{Zeitreihenmanagement}
Im Zentrum des betrachteten Segment des Datenbankschemas steht die Tabelle: ZRM\_ZR. 
Sie dient als zentrale Tabelle zur Verwaltung der im Energiemanagementsystem gespeicherten Zeitreihen und weist jeder Zeitreihe eine eindeutige ID zu.
Neben der ID enthält die Tabelle die Bezeichnung einer Zeitreihe, den Typ einer Zeitreihe  welcher zwischen M-Messwert, P-Prognose 
und F-Fahrplan variieren kann und einigen weiteren Spalten.

Die Tabelle steht in N:1 Beziehung zur Tabelle ZRM\_AZ welche Informationen bezüglich Abtastzeiten von Zeitreihen speichert. 
In ZRM\_AZ wird unter anderem die ID, Zeitzone, die Bezeichnung, der Gültigkeitszeitraum, das Zeitintervall aufgelöst bis zur Sekunde und das 
Offset aufgelöst bis zu Minute einer Abtastzeit gespeichert.

Des weiteren steht die Tabelle ZRM\_ZR in N:1 Relation zur Tabelle: ZRM\_ME welche Informationen bezüglich Maßeinheiten von Zeitreihen speichert.
In der Tabelle sind neben der ID der Maßeinheit auch deren konkrete Einheit, die Bezeichnung der Größe und weitere Informationen Spalten. 
Eine wichtige Rolle haben die Spalten: ME\_ISBASISEINHEIT welche angibt ob die Maßeinheit eine Basiseinheit darstellt, und wenn dies nicht gilt 
ME\_UFAKTOR welche den Umrechnungsfaktor zur entsprechenden Basiseinheit hält.

Über die N:M Relation zwischen ZRM\_ZR und ZRM\_ZRINFO über ZRM\_ZR\_ZRINFO wird eine Zuweisung von in ZRM\_ZRINFO angelegten Infofeldern 
mit einem in ZRM\_ZR\_ZRINFO vergebenen Wert zu einer Zeitreihe realisiert. 
Die hier abgebildeten Infofelder können als Key-Value Paare für Zeitreihen betrachtet werden, wobei ZRM\_ZRINFO.ZRINFO\_BEZ den Key und 
ZRM\_ZR\_ZRINFO.ZR\_INFO\_TEXT den Wert realisiert.

Über die N:1 Beziehung zwischen ZRM\_ZR und DAT\_MANDANT werden alle Zeitreihen mit einem Mandanten Verknüpft. 
Mandanten können ganze Organisationen oder Teile von Organisationenen repräsentieren und dienen der Zuweisung von Zuständigen für die Zeitreihen. 
So wie die anderen Tabellen enthält diese Tabelle Spalten zur Überwachung von Änderungen und dem Anlegen.

\subsubsection{Zeitreihenwerte}

Die Zeitreihenwerte einer in EDM-EMS-Prophet® gespeicherten Zeitreihe werden in der ZRM\_ZW1 gespeichert, welche in einer N:1 Beziehung zur Tabelle ZRM\_ZR steht.
In der Tabelle werden nur äquidistante Werte, also Werte welche im gleichen Zeitintervall abgetastet werden, abgelegt.
Die drei zentralen Spalten der Tabelle sind ZRM\_ZRID welche die relation zur Zeitreihe realisiert, ZW1\_UT welche den Endzeitstempel eines Werts in UTC hält 
und ZW1\_Wert welche den Zeitreihenwert in der Basiseinheit hält.
Hat man Beispielsweise eine Zeitreihe welche die Messwerte des Energievebrauchs einer Anlage in kWh speichert entspricht der Wert in ZW1\_Wert zum Endzeitstempel 
ZW1\_UT der Energiemenge die in dieser Anlage seit vom vorherhigen Endzeitstempel verbraucht wurde. 
Die Werte in ZRM\_ZW1 stellen somit im Falle von Zeitreihen die die Messergebnisse von Energieströme speichern, Aggregate des Energiestroms über den in ZRM\_AZ 
festgelegten Abtastzeitraum dar.
ZW1\_Version und ZW1\_UPDATE dienen der Versionierung der Werte, da die Zeitreihenwerte angepasst werden können.
Somit ist die Grundlegende Funktionalität Energieströme zu erfassen und zu speichern gegeben. 
Um Bilanzräume abzubilden fehlen folglich Tabellen welche erfasste Zeitreihen in ein durch das Bilanzraumkonzept erarbeite Zusammenhänge bringen und Berechnungnen 
über diese Zeitreihen durchführen.