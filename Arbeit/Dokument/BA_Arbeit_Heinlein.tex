\documentclass[a4paper,10pt,twoside]{report}

% Packages for enhancing LaTeX capabilities
\usepackage[a4paper, inner=4cm, outer=3cm, top=2.5cm, bottom=2.5cm]{geometry}
\usepackage{fontspec}
\usepackage{setspace}
\usepackage{fancyhdr}
\usepackage{graphicx}
\usepackage[backend=biber,style=apa]{biblatex}
\usepackage{csquotes}
\usepackage[ngerman]{babel}
\usepackage{float}
\usepackage{amsmath}
\usepackage{fancyhdr}


\raggedbottom


\setmainfont{Arial}
\setlength{\baselineskip}{1.2\baselineskip}
\pagenumbering{roman}
\addbibresource{../../Ressourcen/Bibliographie/ba_literatur.bib}


\begin{document}

% Titlepage
\begin{titlepage}
  \hspace*{\fill}
  \includegraphics[width=0.4\textwidth,keepaspectratio]{../../Ressourcen/Bilder/FHE-AI_Logo.png}

  \centering
  \vspace*{1cm}

  \vspace{0.5cm}

  {\Huge\textbf{Bachelorarbeit in der Angewandten Informatik}}
  \\
  Registriernummer: AI-2024-BA-030

  \vspace{1.5cm}

  {\large\textbf{Konzeption und Entwicklung einer datenbankseitigen Abbildung von frei definierbaren Bilanzräumen
      im Zusammenhang mit dem Energiemanagementsystem EMS-EDM PROPHET® nach ISO 50001.}}

  \vspace{1.5cm}

  {\Large\textbf{Fabian Heinlein}}
  \\ in Kooperation mit dem Fraunhofer Institut Angewandte Systemtechnik (IOSB-AST)

  \vspace{1cm}

  \Large Abgabedatum: 28.02.2024

  \vspace{1cm}

  Prof. Dr. Marcel Spehr \\
  Sven Möller

  \vfill

\end{titlepage}

\setcounter{page}{1}



\chapter*{Kurzfassung}
\addcontentsline{toc}{chapter}{Kurzfassung}
\chapter*{Abstract}
\addcontentsline{toc}{chapter}{Abstract}
\chapter*{Vorwort}
\addcontentsline{toc}{chapter}{Vorwort}
\tableofcontents

%%%%%%%%%%%%%%%%%%%%%%%%%% Einleitung %%%%%%%%%%%%%%%%%%%%%%%%%%
\chapter{Einleitung}
\pagenumbering{arabic}
\setcounter{page}{1}

\section{Hintergrund und Motivation}
Angesichts wachsender Umweltbelastungen und der Notwendigkeit nachhaltiger Praktiken spielt das Energiemanagement eine immer bedeutendere Rolle.
Diese Arbeit untersucht die Entwicklung einer datenbankseitigen Lösung zur Abbildung frei definierbarer Bilanzräume im Energiemanagementsystem
EMS-EDM PROPHET® nach DIN EN ISO 50001:2018-12.
Sie wird durch das Potenzial, die energiebezogene Leistung und Energieeffizienz von Organisationen durch die Erfüllung ausgewählter Kriterien 
der DIN EN ISO 50001:2018-12 zu verbessern, motiviert. 
Bilanzräume stellen das zentrale Konzept der Arbeit dar und werden im Rahmen dieser als Einheiten betrachtet, die zur digitalen Abbildung 
von Organisationsstrukturen im Energiemanagement und als administrative Grenze zur Bilanzierungsrechnung dienen.
Die Adressierung der Arbeit auf die freie definierbare Gestaltung der Bilanzräume soll eine Möglichkeit bieten, der Diversität von Organisationen gerecht zu werden
und einen Einsatz der Forschungsergebnisse in Organisationen mit dem EDM-EMS-Prophet® ermöglichen.
Die Untersuchung soll zur Weiterentwicklung nachhaltiger Energiemanagementpraktiken beitragen und Einblicke in die Integration
technischer Lösungen in bestehende Systeme bieten.

Ein wesentlicher Fokus dieser Arbeit liegt auf der DIN EN ISO 50001:2018-12, einer Norm der Internationalen Organisation für Normung (ISO),
die Anforderungen an Energiemanagementsysteme festlegt. Diese Norm ist universell einsetzbar, unabhängig von Größe, Art oder Standort der Organisation (\cite[S. 10]{DIN50001.2018}),
und dient der fortlaufenden Verbesserung der energiebezogenen Leistung. (\cite[S. 7]{DIN50001.2018}).
Um die Anforderungen der DIN EN ISO 50001:2018-12 zu erfüllen, müssen Organisationen den kontinuierlichen Fortschritt ihrer energiebezogenen Leistung nachweisen, wobei
die Norm keine spezifischen Zielniveaus vorgibt. (\cite[S. 10]{DIN50001.2018}).

Die Umsetzung der DIN EN ISO 50001:2018-12 in Organisationen bringt sowohl operationale als auch organisatorische Herausforderungen mit sich [S. 11](\cite{Marimon.2017}).
Dennoch lag im Jahr 2023 in 24.924 Organisationen weltweit ein Zertifikat nach DIN EN ISO 50001:2018-12 vor (\cite{InternationalOrganizationforStandardization.2023}).
Dies ist bemerkenswert, da die Erfüllung der Normanforderungen voraussichtlich etwa 60 \% des globalen Energieverbrauchs beeinflussen
kann (\cite{InternationalOrganizationforStandardization.2011}, zitiert nach \cite[S. 1]{Marimon.2017}). Darüber hinaus entstehen für Organisationen durch
die Einführung der Norm signifikante Vorteile.

Zum einen können nach Aussagen der DIN EN ISO 50001:2018-12 (2018, S. 9) ökonomische Vorteile wie Energieeinsparungen erzielt werden, wodurch Organisationen einen Wettbewerbsvorteil
aufgrund sinkender Energiekosten erlangen können. Zum anderen ergeben sich operationale Vorteile wie eine gesteigerte Produktivität, verbesserte Qualität
und ein strukturierter Ansatz zur Prozessoptimierung (\cite{Marimon.2017}). Des Weiteren kann die Umsetzung der DIN EN ISO 50001:2018-12 dazu beitragen, die allgemeinen
Klimaschutzziele zu erreichen (\cite{DIN50001.2018}). Dies unterstreicht die gesellschaftliche Bedeutung der Norm, insbesondere angesichts der Herausforderungen
des Klimawandels.

Die Umsetzung der DIN EN ISO 50001:2018-12 basiert auf dem PDCA-Zyklus (Plan, Do, Check, Act), der Organisationen einen strukturierten Rahmen für die fortlaufende
Verbesserung der energiebezogenen Leistung bieten soll (\cite[S. 7f.]{DIN50001.2018}).
Während die Norm in erster Linie Anforderungen auf Managementebene formuliert, verweist sie auch auf technische Normen wie die E DIN ISO 50006:2024-07, die unter anderem
spezifische Anforderungen an Energieleistungskennzahlen und energetische Ausgangsbasen definiert (\cite{DIN50006.2024,DIN50001.2018}).


\section{Problemstellung}
\subsection{Problembeschreibung}
\textbf{Forschungsfrage:} "Welche strukturellen Erweiterungen und Anpassungen müssen auf Datenbankebene in EMS-EDM PROPHET® konzipiert und implementiert 
werden, um eine frei definierbare Abbildung von Bilanzräumen zu ermöglichen, die Organisationen bei der Erfüllung der ISO 50001 unterstützt?"

Die breite Anwendbarkeit der in der DIN EN ISO 50001:2018-12 gestellten Anforderungen auf Organisationen führt zu Anforderungen an die Abbildbarkeit von 
frei definierbaren energiebezogenen Organisationsstrukturen im Energiemanagementsystem.

Ein Teilaspekt zur Umsetzung der DIN EN ISO 50001:2018-12 im Rahmen der Energiebilanzierung ist die Abbildung von energiebezogenen Bereichen in 
Organisationen im Energiemanagementsystem und soll im Rahmen dieser Arbeit betrachtet werden.

Die aktuelle Datenbankstruktur von EMS-EDM Prophet® steht vor dem Problem, frei definierbare Bilanzräume abzubilden.
Um dieses Problem zu lösen, sind strukturelle Änderungen und Erweiterungen der Datenbank notwendig.
Somit besteht das zentrale Problem dieser Arbeit darin, ein System zur Abbildung frei definierbarer Bilanzräume auf Datenbankebene unter Berücksichtigung 
der von der DIN EN ISO 50001:2018-12 gestellten Anforderungen an Bilanzräume und damit verbundene Themenkomplexe in EMS-EDM Prophet® zu konzipieren und 
zu implementieren.

Die Problemlösung umfasst alle Aspekte, die auf Grundlage der Vorgaben der Norm sowie praktischer Gegebenheiten konzipiert und auf Datenbankebene 
umgesetzt werden müssen, um EMS-EDM PROPHET® so zu erweitern, dass das System in der Lage ist, Organisationen bei der Erfüllung der ISO 50001 zu 
unterstützen. Dies gilt insbesondere für Anforderungen, die durch die Abbildung von Bilanzräumen adressiert werden können.

Aufgrund der Anwendbarkeit der DIN EN ISO 50001:2018-12 auf alle Organisationen ist die freie Definierbarkeit der Bilanzräume ein Qualitätskriterium des zu 
entwerfenden Systems und spielt bei der Beantwortung der Forschungsfrage eine zentrale Rolle.
Die breite Anwendbarkeit der Norm impliziert außerdem die Notwendigkeit, praktische Herausforderungen beim Einsatz der Lösung zu berücksichtigen und 
Anwendungsgebiete des entworfenen Konzepts zu betrachten, um der praktischen Relevanz dieser Arbeit gerecht zu werden.

\subsection{Praktische Relevanz des Problemraums} 
Das beschriebene Problem weist eine praktische Relevanz auf, da es die Herausforderungen der DIN EN ISO 50001:2018-12
im Energiemanagement von Organisationen adressiert.
Die bestehenden Anforderungen der DIN EN ISO 50001:2018-12 und der aktuelle Zustand von EMS-EDM Prophet® stellen praxisnahe Qualitätskriterien an die Abbildung von Bilanzräumen.
Eine Herausforderungen besteht darin, ein Datenbankmodell zu entwickeln, das diese Anforderungen erfüllt und gleichzeitig praxisnah und umsetzbar ist.
Die Berücksichtigung von aus der Praxis abgeleiteten Anforderungen ist dabei unerlässlich.
Dies verdeutlicht die Notwendigkeit einer Methodik, die sowohl theoretische als auch praktische Aspekte integriert.
Die Integration der Lösung in EMS-EDM Prophet® stellt sicher, dass sie in bestehenden Organisationen nutzbar ist und deren Energiemanagement unterstützt.

\subsection{Wissenschaftliche Relevanz des Problemraums} 
Die Problemstellung weist eine wissenschaftliche Relevanz auf, da im Zuge der Erarbeitung einer Lösung Methoden des Datenmanagements im Kontext der 
Modellierung von Energiebilanzräumen angewandt werden.
Dabei werden die in EMS-EDM Prophet® bestehenden Methoden um neue Ansätze zur Modellierung von Bilanzräumen erweitert.
Diese Erweiterungen tragen zur wissenschaftlichen Diskussion über Datenmanagementstrategien im Energiemanagement bei und bieten neue Perspektiven für die 
Integration von Bilanzräumen in datenbankbasierte Systeme.
Darüber hinaus fördert die Arbeit den interdisziplinären Austausch zwischen den Bereichen Energiemanagement und Datenbankmodellierung, indem sie 
theoretische Konzepte mit praktischen Anwendungen verknüpft.
Die entwickelten methodischen Ansätze und Modelle können als Grundlage für zukünftige wissenschaftliche Untersuchungen dienen und die Weiterentwicklung 
von Energiemanagementsystemen unterstützen.


\section{Ziel der Arbeit}

Das Ziel dieser Arbeit ist die Konzeption, Implementation und Evaluation eines Prototyps, der durch strukturelle Anpassungen und 
Erweiterungen des EMS-EDM Prophet® die Abbildung frei definierbarer Bilanzräume ermöglicht. Der Prototyp soll einen Mehrwert zur 
Erfüllung der Anforderungen der DIN EN ISO 50001:2018-12 bieten und in Organisationen des tertiären Wirtschaftssektors, die EMS-EDM Prophet® nutzen, 
anwendbar sein. 
Der erarbeitete Prototyp soll auf den theoretischen Grundlagen des Energiemanagements basieren und bewährte Ansätze 
aus dem Forschungsgebiet Energiebilanzierung Datenbankseitig abbilden.
Des weiteren soll der Prototyp einen Mehrwert zur Erfüllung von ISO 50001 Anforderungen hinsichtlich der bestimmung von wesentlichen Energieeinsätzen und 
der Bewertung der energiebezogenen Leistung bieten und den Umfang der im Rahmen des EMS-EDM Prophet® addressierten ISO 50001 Anforderungen erweitern.
Außerdem soll der Prototyp praktische Herausforderungen und Gegebenheiten im Anwendungskontext berücksichtigen.


Zur Evaluation des Prototyps soll die Bilanzraumstruktur der Organisation: Fraunhofer IOSB-AST in Ilmenau erarbeitet und im entworfenen Prototyp abgebildet werden.
Der angewendete Prototyp soll hinsichtlich der Erfüllung von DIN EN ISO 50001:2018-12 Anforderungen, und der korrekten Abbildung der theoretischen Grundlagen 
auf qualitative und quantitative Qualitätskriterien evaluiert werden.
Im Rahmen der evaluierung soll die freie Definierbarkeit im Anwendungsgebiet: Organisationen des tertiären Wirtschaftssektors betrachtet werden. 


\section{Aufbau der Arbeit}
Diese Arbeit ist so konzipiert, dass Sie die theoretischen Grundlagen des Problemraums erfasst und Nutzen sowie Herausforderungen im Anwendungsgebiet: 
EMS nach DIN EN ISO 50001:2018-12 erarbeitet. 
Basierend auf den theoretischen Grundlagen im Anwendungsbereich und den bestehenden Methoden und Ansätzen des Datenmanagements 
in EMS-EDM Prophet® wird eine Lösung der Forschungsfrage auf Datenbankebene des Energiemanagementsystems konzipiert, 
implementiert und evaluiert.
Der Aufbau der Arbeit umfasst drei Hauptabschnitte: die theoretischen Grundlagen und der Stand der Wissenschaft, die Konzeption und Implementation, und die 
Evaluation.

\begin{enumerate}
    \item \textbf{Theoretische Grundlagen und Stand der Wissenschaft}
    
    Die praxisnahe Problemstellung erfordert eine anwendungsorientierte Forschung unter Berücksichtigung der Interdisziplinarität. 
    Im theoretischen Teil der Arbeit werden zwei Themenbereiche betrachtet: Grundlagen der Energiebilanzierung unter Nutzung von Bilanzräumen und  
    Energiemanagementsysteme nach DIN EN ISO 50001:2018-12. In beiden Tehemenbereichen findet die erarbeitung der Grundlagen unter beachtung des Anwendungsgebiets: 
    Organisation im tertiären Wirtschaftssektor statt. 
    
    Für die Erarbeitung der theoretischen Grundlagen des Energiemanagements werden im ersten Hauptabschnitt der Arbeit die DIN EN ISO 50001:2018-12, 
    damit verbundene Normen und Basiswissen aus für den Problemraum relevanter Fachliteratur analysiert.
    Außerdem werden wissenschaftliche Arbeiten aus verwandten Problemräumen analysiert und in den Kontext dieser Arbeit gesetzt.
    Auf dieser Basis werden theoretische Konzepte und Anforderungen aus dem Problemraum abgeleitet, die für die Lösung der Forschungsfrage relevant sind.
    
    Der erste Hauptabschnitt der Arbeit hat somit eine zentrale Bedeutung zum erreichen der Interdisziplinarität der Forschung.
    Die umfangreiche erarbeitung von Konzepten des Energiemanagements, Anforderungen von Anforderungen der ISO 50001 und den Einsatzmöglichkeiten 
    von Bilanzräumen zur praxisnahen erfüllung dieser Anforderungen auf basis der Konzepte stellen eine detaillierte Analyse des Anwendungsbereichs dieser 
    Forschung dar.
    Diese Detaillierte Analyse, ohne technische Perspektive der Datenbankmodellierung, ist notwendig um der Interdisziplinarität des Problemraums aus Sicht des 
    Energiemanagements und der Energiebilanzierung gerecht zu werden. 
    Die erarbeiteten Grundlagen des Anwendungsbereichs: Energiemanagent wird im nächsten Hauptabschnitt, der Konzeption und Implementation, 
    aufgegriffen und aus einer technischen Sicht des Datenbankmanagements betrachtet.

    \item \textbf{Konzeption und Implementation des Prototyps}

    Basierend auf den Forschungsergebnissen des theoretischen Teils der Arbeit wird im zweiten Kapitel der Arbeit eine Lösung für den Problemraum
    konzipiert und implementiert.
    Um der Interdisziplinarität aus Sicht der Datenbankmodellierung gerecht zu werden, wird der IST-Zustand des EMS-EDM Prophet® analysiert und es werden 
    bereits bestehende Ansätze der Datenbankmodellierung die den Problemraum addressieren aufgezeigt.
    Unter berücksichtigung der aufgezeigten Ansätze wird der Prototyp zur Problemlösung konzipiert und EMS-EDM Prophet® im implementiert.
    Im Zuge dessen ist die konzeption und integration von Ansätzen des Datenbankmanagements, die noch nicht im EMS-EDM Prophet® bestehen 
    notwendig.
    Das Konzept addressiert die im ersten Hauptabschnitt erarbeiteten Erkenntnisse des Anwendungsbereichs: Energiemanagement und Energiebilanzierung 
    und stellt eine Datenbankseitige abbildung der Grundsätze unter den erarbeiteten Anforderungen der ISO 50001 im kontext von Organisationen des 
    tertiären Wirtschaftssektors zur verfügung.
    
    \item \textbf{Evalutation des Prototyps}
    
    Im dritten Hauptabschnitt der Arbeit wird der entworfene Prototyp evaluiert.
    Im Zuge dessen wird die Bilanzraumstruktur des Fraunhofer IOSB-AST in Ilmenau erarbeitet und im entworfenen Prototyp abgebildet.
    Anhand dieses praktischen Beispiels wird getestet wie korrekt die im ersten Hauptabschnitt erläuterten Methoden des Energiemanagements und der Energiebilanzierung 
    umgesetzt wurden, und wie der Prototyp in der Praxis bei der Erfüllung von ISO 50001 Anforderungen Organisationen unterstützen kann. 
    Dieser Abschnitt der Evaluation findet unter Nutzung quantitativer Qualitätskriterien statt.

    Des weiteren wird grundsätzlich betrachtet, wie im Rahmen der Interdisziplinarität die im ersten Hauptabschnitt erarbeiteten Erkenntnisse aus Sicht 
    des Energiemanagements technisch durch den konzipierten und implementierten Prototyp abgebildet wurde.
    Dabei wird auch analysiert ob und wie die im ersten Hauptabschnitt erörterten Anforderungen der ISO 50001 auf Datenbankebene umgesetzt wurden sind.
    Dieser Abschnitt der Evaluation findet unter Nutzung qualitativer Qualitätskriterien statt.

\end{enumerate}


%%%%%%%%%%%%%%%%%%%%%%%%%% Theoretische Grundlagen %%%%%%%%%%%%%%%%%%%%%%%%%%

\chapter{Stand der Forschung und Theoretische Grundlagen}
\section{Bilanzräume im Kontext der Energiebilanzierung}
\subsection{Grundlagen der Bilanzierung}

%%%%%%%%%%%%%%%%%%%%%%%%%%%%%%%%%%%%%%%%%%%%%%%%%%%%%%%%%%%%%%%%%%%%%%%%%%%%%%%%%%%%%%%%%%%%%%%%%%%%%%%%%%%%%%%%%%%%%%%%%%%%%%%%%%%%%%%%%%%%%%%%%%%%%%%%%%%%%%%%%%%
\subsubsection{Anwendungskontext der Bilanzierung}

Bilanzierung ist ein Konzept, welches in unterschiedlichen Einsatzbereichen Verwendung findet. Diese Forschung befasst sich mit Bilanzräumen im Kontext des 
Energiemanagements nach DIN EN ISO 50001:2018-12. Die Norm setzt den Schwerpunkt auf die fortlaufende Verbesserung der energiebezogenen Leistung 
(\cite[Kapitel 0.2]{DIN50001.2018}). Somit kann der Kontext auf die Bilanzierung energetischer Größen eingegrenzt werden.
Auch die Festlegung auf Organisationen des tertiären Wirtschaftssektors hat Auswirkungen auf die Betrachtungsweise der Bilanzierung. Denn in Organisationen 
mit immateriellen Dienstleistungen spielt die Gebäudeenergie eine vorrangige Rolle zur Verbesserung der energiebezogenen Leistung (\cite[S. 3]{Fichera.2020}).
Dies lässt sich beispielhaft an der Abbildung \eqref{fig:Energieverbrauch_Wärme_DE} darstellen.

\begin{figure}[H]
    \centering
    \includegraphics[width=0.85\textwidth]{../../Ressourcen/Abbildungen/Energieverbrauch_für_Wärmezweck_DE.jpg}
    \caption{Energieverbrauch für den Wärmezweck in Deutschland (Dargestellt von AGEB (2024))}
    \label{fig:Energieverbrauch_Wärme_DE}
\end{figure}

Die Abbildung \ref{fig:Energieverbrauch_Wärme_DE} zeigt den Energieverbrauch für Wärmezwecke in Deutschland im Jahr 2023, aufgeschlüsselt nach Sektoren. 
Während der industrielle Sektor einen hohen Anteil an prozessbezogener Wärme aufweist, 
spielt im Dienstleistungssektor die Raumwärme eine dominante Rolle.
Diese Statistik bekräftigt die Aussage von Fichera (2020, S. 3), dass bei der Verbesserung der energiebezogenen Leistung in Organisationen des tertiären 
Wirtschaftssektors energiebezogene Prozesse und Technologien im Gegensatz zur Gebäudeenergie eine untergeordnete Bedeutung haben.

Im Rahmen der Bestimmung des Gesamtenergiebedarfs eines Gebäudes über den Lebenszyklus wird vor allem der Gebäudebetrieb betrachtet (\cite[S. 133]{Musall.2015}).
Die sogenannte Graue Energie wird üblicherweise als kumulierter, nicht erneuerbarer Primärenergieaufwand beschrieben, der alle vor- und nachgelagerten Prozesse 
der verwendeten Baustoffe und Materialien sowie der technischen Anlagen umfasst (\cite[S. 133]{Musall.2015}). Da die DIN EN ISO 50001:2018-12 auf die fortlaufende 
Verbesserung der energiebezogenen Leistung abzielt und die Graue Energie konstant ist, wird diese im Rahmen dieser Arbeit nicht betrachtet.
Folglich ist der Kontext der Bilanzierung in dieser Forschungsarbeit der Energieverbrauchs im Rahmen des Gebäudebetriebs von Organisationen 
des tertiären Wirtschaftssektors.

%%%%%%%%%%%%%%%%%%%%%%%%%%%%%%%%%%%%%%%%%%%%%%%%%%%%%%%%%%%%%%%%%%%%%%%%%%%%%%%%%%%%%%%%%%%%%%%%%%%%%%%%%%%%%%%%%%%%%%%%%%%%%%%%%%%%%%%%%%%%%%%%%%%%%%%%%%%%%%%%%%%

\subsubsection{Konzept}
Im Rahmen des beschriebenen Kontexts rückt die verfahrenstechnische Perspektive der Bilanzierung in den Fokus. 
So wird die Bilanzierung in der Verfahrenstechnik nach Rönsch (2015, S. 66) in drei Bilanzgleichungen unterteilt: die Massenbilanz, 
die Energiebilanz und die Impulsbilanz.
Da der Kontext der Bilanzierung sich auf den Energieverbrauch bezieht hat insbesondere die Energiebilanz eine hohe Relevanz.
Die Energiebilanz beruht auf dem Energieerhaltungssatz (\cite[S. 66]{Rönsch.2015}), der das Prinzip der Erhaltung der Energie ausdrückt 
(\cite[S. 57]{Baehr.1966}). Der Energieerhaltungssatz bezieht sich auf alle Erscheinungsformen, in denen Energie auftritt, und besagt, dass es 
unmöglich ist, Energie zu erzeugen oder zu vernichten (\cite[S. 57]{Baehr.1966}).
Für zu bilanzierende Systeme bedeutet dies, dass die Energie in einem abgeschlossenen, adiabaten System über die Zeit konstant ist 
(\cite[S. 66]{Rönsch.2015}). Adiabat bedeutet in diesem Kontext, dass das System keine Wärme mit seiner Umgebung austauscht (\cite[S. 66]{Rönsch.2015}).

Für Systeme, die in der Lage sind, Energie zu speichern, impliziert dies nach Rönsch (2015, S. 66f.), dass die darin gespeicherte Energie gleich der 
Differenz aus ein- und austretenden Energieströmen ist.
Für offene, nicht-adiabate Systeme ohne Speicherfähigkeit gilt, dass die Differenz der ein- und austretenden Energieströme null ist (\cite[S. 66f.]{Rönsch.2015}).
Das von Rönsch (2015, S. 66f.) beschriebene Verhalten eines Systems bezüglich der Energiespeicherung lässt sich mathematisch vereinfacht mit der Gleichung 
\eqref{energiebilanzierungsgleichung_Rönsch} darstellen:

\begin{equation}
E_{\text{gespeichert}} = \sum E_{\text{eingang}} - \sum E_{\text{ausgang}}
\label{energiebilanzierungsgleichung_Rönsch}
\end{equation}

\begin{description}
    \item \(E_{\text{gespeichert}}\): Im System gespeicherte Energie.
    \item \(E_{\text{eingang}}\): Energie eines eintretenden Energiestroms.
    \item \(E_{\text{ausgang}}\): Energie eines austretenden Energiestroms.
    \item Für offene, nicht-adiabate Systeme ohne Energiespeicher gilt:
    \[
    E_{\text{gespeichert}} = 0
    \]
    \item In diesem Fall ist die zugeführte Energie gleich der abgegebenen Energie:
    \[
    \sum E_{\text{eingang}} = \sum E_{\text{ausgang}}
    \]
\end{description}

Ein weiterer Ansatz aus verfahrenstechnischer Perspektive wird von der von Ahrendts (2014, Kapitel 1.5) aufgestellten Bilanzgleichung im Kontext der Thermodynamik 
adressiert.
Der Gleichung liegt der Fakt zugrunde, dass sich für jede mengenartige Zustandsgröße, die über die Grenze eines Systems transportiert wird, eine Bilanz aufstellen lässt 
(\cite[Kapitel 1.5]{Ahrendts.2014}).
Diese Bilanz umfasst ein- und austretende Ströme sowie im System enthaltene Energiequellen und -senken und ermittelt die Geschwindigkeit der Änderung des Bestands der 
zu bilanzierenden Zustandsgröße im System (\cite[Kapitel 1.5]{Ahrendts.2014}).

Die von Ahrendts (2014, Kapitel 1.5) aufgestellte Bilanzgleichung wird in den Formeln \eqref{BilanzierungsgleichungAhrendt} und 
\eqref{BilanzierungsgleichungAhrendtStrom} dargestellt.

\begin{equation}
    \frac{dX_{\text{j}}}{d\tau} = (\sum \dot{X}_{\text{j,e}} - \sum \dot{X}_{\text{j,a}}) + (\dot{X}_{\text{j,Quell}} - \dot{X}_{\text{j,Senk}})
    \label{BilanzierungsgleichungAhrendt}
\end{equation}

\begin{description}
    \item \(X_{\text{j}}\): Zustandsgröße.
    \item \(\tau\): Zeitintervall.
    \item \(X_{\text{j,e}}\): Über die Systemgrenze zufließende Zustandsgröße.
    \item \(X_{\text{j,a}}\): Über die Systemgrenze abfließende Zustandsgröße.
    \item \(X_{\text{j,Quell}}\): Quellen der Zustandsgröße im System.
    \item \(X_{\text{j,Senk}}\): Senken der Zustandsgröße im System.
\end{description}


\begin{equation}
    \dot{X}_{\text{j}} = \lim_{\Delta\tau \to 0} \Delta X_{\text{j}}/ \Delta\tau
    \label{BilanzierungsgleichungAhrendtStrom}
\end{equation}

\begin{description}
    \item \(X_{\text{j}}\): Zustandsgröße.
    \item \(\Delta X_{\text{j}}\): Menge der Größe \(X_{\text{j}}\) im Zeitintervall \(\Delta \tau\).
    \item \(\Delta \tau\): Zeitintervall.
\end{description}


 

Die Gleichung \eqref{BilanzierungsgleichungAhrendt} in Verbindung mit \eqref{BilanzierungsgleichungAhrendtStrom} beschreibt die Geschwindigkeit der Änderung des Bestands der Größe
\(X_{\text{j}}\) als Summe der Differenzen zwischen den über die Systemgrenze zu- und abfließenden Strömen der Zustandsgröße
\(X_{\text{j}}\) sowie den Quell- und Senkströmen der Zustandsgröße \(X_{\text{j}}\) innerhalb des Systems.  
Zur angemessenen Anwendung der Gleichung \eqref{BilanzierungsgleichungAhrendt} stellt sich die Frage nach einer angemessenen Zustandsgröße.
In einem thermodynamischen System wird der augenblickliche Zustand durch die Zustandsgrößen beschrieben, wobei diese in intensive und extensive Zustandsgrößen 
unterschieden werden (\cite[S. 66]{Konstantin.2023}). Die innere Energie U mit der Basiseinheit Joule ist eine extensive Zustandsgröße 
(\cite[S. 65]{Konstantin.2023}) und rückt in den Fokus, da es sich um eine Zustandsgröße der Energie handelt.
Folglich können zu- und abfließende Energieströme mit Energiequellen und -senken bilanziert werden.
Im Rahmen der Formel \eqref{BilanzierungsgleichungAhrendt} wird der Strom einer Zustandsgröße \(X_{\text{j}}\) in Gleichung \eqref{BilanzierungsgleichungAhrendtStrom} definiert.
Der Strom einer Zustandsgröße wird als Menge der Zustandsgröße in einem infinitesimal kleinen Zeitintervall definiert, welches im Grenzwert gegen 0 geht.
Folglich wird ein Strom von Ahrendts (2014) als Menge einer Zustandsgröße zu einem bestimmten Zeitpunkt definiert.

Die Gleichungen \eqref{energiebilanzierungsgleichung_Rönsch} und \eqref{BilanzierungsgleichungAhrendt} formulieren in Verbindung mit 
\eqref{BilanzierungsgleichungAhrendtStrom} eine grundlegende und zugleich vereinfachte mathematische Beschreibung einer Bilanzierung im Kontext der Thermodynamik und 
Verfahrenstechnik. Sie bilden die Basis für die Beschreibung der grundlegenden Struktur einer Bilanz.
Im Folgenden werden die in \eqref{energiebilanzierungsgleichung_Rönsch} und \eqref{BilanzierungsgleichungAhrendt} mit \eqref{BilanzierungsgleichungAhrendtStrom} 
beschriebenen Bestandteile einer Bilanz zur erarbeitung eines Bilanzraumkonzepts im Anwendungskontext des Problemraums analysiert.
\subsection{Definition von Bilanzräumen}

% TODO: Definition Haupmerkmale, Kontext SEU-Bilanz
% TODO: Umformulieren

Bei der Wahl einer Zustandsgröße haben neben der Zweckmäßigkeit auch die Messinfrastruktur in Organisationen eine Relevanz.
Denn die Gleichung \eqref{BilanzierungsgleichungAhrendt} beschreibt das Verhalten der Zustandsgröße in Abhängigkeit von den zu- und abfließenden Zustandsgrößen und den 
Quellen und Senken der Zustandsgröße. Wenn die Ströme, Quellen und Senken der Zustandsgröße nicht vollständig Erfasst werden ist somit auch keine korrekte Bilanzierung 
möglich. 
Die DIN EN ISO 50001:2018-12 (2018, Kapitel 6.6, A.6.6) stellt Anforderungen und Qualitätskriterien an die Datensammlung in Organisationen.
Diese verpflichtet Organisationen dazu Hauptmerkmale ihrer Tätigkeiten, die sich auf die energiebezogene Leistung auswirken zu identifizieren, und diese in geplanten 
Zeitabständen zu messen, überwachen und analysieren (\cite[S. 23]{DIN50001.2018}).
Teil der zu erfassenden Hauptmerkmale sind relevante Variablen bezüglich wesentlicher Energieeinsätze, den Energieverbrauch bezüglich wesentlicher Einsätze 
und der Organisation und betriebliche Kriterien bezüglich wesentlicher Energieeinsätze(\cite[S. 23]{DIN50001.2018}).
Die komplexität der Umsetzung ist dabei nicht vorgeschrieben und kann von einfachen Zählwerten bis hin zu umfangreichen Werten aus Überwachungs- und Messystemen mit 
Softwareandwendung reichen (\cite[S. 36]{DIN50001.2018}).

\section{Energiemanagement nach ISO 50001}

\input{../Kapitel/02_Theoretische_Grundlagen/02_Bilanzräume_im_Energiemanagement_nach_ISO_50001/01_Energieleistungskennzahlen.tex}
\subsection{Identifikation wesentlicher Energieeinsätze}

\subsubsection{Analyse und Unterscheidung von Energieeinsätzen}

Die DIN EN ISO 50001:2018-12 verpflichtet Organisationen im Rahmen der Planungsphase des PDCA-Zyklus zur Identifikation von wesentlichen Energieeinsätzen auf Grundlage 
der vorher durchgeführten Datenanalyse (\cite[S. 25]{DIN50001.2018}).
Die Norm definiert einen Energieeinsatz als Anwendung von Energie zum Beispiel für Energiedienstleistungen wie Lüftung oder Heizung, und bezeichnet den Begriff mitunter als 
Endnutzung von Energie (\cite[Kapitel 3.5.4]{DIN50001.2018}). 
Der Energieeinsatz ergibt sich aus dem Produkt des spezifischen Energieeinsatzes und der Menge der Nachgefragten Energiedienstleistungen (vgl. Gleichung \eqref{EnergieeinsatzMiller}) (\cite[S. 120]{Miller.2016}).
Der Spezifische Energieeinsatz ergibt sich aus dem Kehrwert der Energieeffizienz (vgl. Gleichung \eqref{EffizienzgleichungMiller}) (\cite[S. 120]{Miller.2016}).
\begin{equation}
    \text{Energieeinsatz} := \text{Spezifischer Energieeinsatz} \cdot \text{Menge Energiedienstleistung}
    \label{EnergieeinsatzMiller}
\end{equation}

\begin{equation}
    \text{Spezifischer Energieeinsatz} :=\frac{\text{Aufwand}}{\text{Erreichter Nutzen}}
    \label{SepzifischerEnergieinsatzMiller}
\end{equation}

Betrachtet man beispielsweise eine Heizungsanlage als nutzenseitige Energiedienstleistung im Untersuchungsgegenstand Gebäudezone.
So könnte man den erreichten Nutzen mit der Grundfläche konkretisieren. 
Der Aufwand wird durch den im Bilanzzeitraum anfallenden Nutzenergiebedarf zur befriedigung der Energiedienstleistung: Heizung gemessen.
Der Spezifische Energieeinsatz ist somit der im Bilanzzeitraum entstandene Nutzenergiebedarf zum Heizen pro Quadratmeter. 

Ein wesentlicher Energieeinsatz, auch SEU (en: significant energy use), wird von der Norm als Energieeinsatz der wesentlichen Anteil am Energieverbrauch 
hat und/oder erhebliches Potential für eine Verbesserung der energiebezogenen Leistung bietet definiert (\cite[Kapitel 3.5.6]{DIN50001.2018}). 
SEUs können Anlagen beziehungsweise Standorte, Systeme, Prozesse oder eine Einrichtungen sein (\cite[Kapitel 3.5.6]{DIN50001.2018}).
Zur Definition von Kriterien zur Identifikation von SEUs macht die Norm keine Angaben und verpflichtet die Organisation die die Norm anwendet zur Entscheidung was 
als wesentlicher Energieeinsatz anzusehen ist (\cite[S. 38]{DIN50001.2018}). 
Neben Energieerzeugungsanlagen und Umwandlungsanlagen gibt es Anlagenkategorien für Klimatisierungsanlagen, 
Lüftungsanlagen, Bleuchtungsanlagen sowie Informations- und Kommunikationstechnik (\cite[S. 14]{Hohnhold.2013}).

Eine differenzierte Darstellung der Verbrauchsstrukturen nach Anlagenkategorien beziehungsweise einzelner Anlagen ermöglicht das identifizieren von 
wesentlichen Energieeinsätzen und liefert somit auch Ansatzpunkte zur Verbesserung der Energieeffizienz (\cite{Fink.1997} zitiert nach \cite[S. 8]{Hohnhold.2013}).
Die in Abbildung \eqref{fig:Disagggregation_Bilanzraum_Nutzengrößen} dargestellte Disaggregation eines Bilanzraums nach Nutzengrößen kann bei der Analyse der 
Verbrauchsstrukturen innerhalb eines Untersuchungsgegenstands im Rahmen der Datenanalyse beitragen.



Abbildung \eqref{fig:Disagggregation_Bilanzraum_Nutzengrößen_Beispiel} zeigt Beispielhaft wie die Disagggregation von Bilanzräumen zur Identifikation von 
wesentlichen Energieeinsätzen beitragen kann. 
Der durch die aufwandsseitigen Ressourcen gedeckte Nutzenergiebedarf der bilanzierten Energiedienstleistung kann als absolute Energieleistungskennzahl zur 
Bewertung des Energieeinsatzes für die Energiedienstleistung betrachtet werden.
Der Energieeinsatz kann mit der durch die Bewertungseinheit quantifizierten Energiedienstleistung in relation gesetzt werden um den Spezifischen Energieeinsatz 
(vgl. Gleichung \eqref{SepzifischerEnergieinsatzMiller}) zu ermitteln, welcher als Beziehungszahl kategorisiert werden kann und somit geeignet zur Bewertung der Energieeffizienz ist. 
Die Integration einer Gliederungszahl als EnPI welche den Energieeinsatz eines Bilanzraums einer Energiedienstleistungen in Relation 
zum Energieeinsatz eines Bilanzraums aller Energiedienstleistungen setzt kann (vgl. Gleichung \eqref{Anteil_Gesamtenergieverbrauch}) den Vergleich des Anteils einzeln Energiedienstleistungen 
am Gesamtenergieverbrauch unterstützen.


\begin{equation}
    \text{Anteil Gesamtenergieverbrauch} :=\frac{\text{Energieverbrauch Bilanzraum}}{\text{Gesamtenergieverbrauch}}
    \label{Anteil_Gesamtenergieverbrauch}
\end{equation}

In diesem Beispiel macht die Heizungsanlage des Hauptgebäudes mit einem Energieeinsatz von 18.750 kWh 62,5 \% des Gesamtenergieverbrauchs aus und hat somit einen 
wesentlich Größeren Anteil am Gesamtenergieverbrauch als die Kühlungsanlage des Hauptgebäudes, welche mit einem Energieeinsatz von 3.750 kWh nur 12,5 \% des 
Gesamtenergieverbrauchs ausmacht.
Mit \( 125 \,\frac{\text{kWh}}{\text{m}^2} \) hat die Heizung den höchsten spezifischen Energieeinsatz und somit die geringst Energieeffizienz und die Kühlung 
mit \( 25 \,\frac{\text{kWh}}{\text{m}^2} \) den geringsten spezifischen Energieeinsatz und somit die höchste Energieeffizienz.

\begin{figure}[H]
    \centering
    \includegraphics[width=0.7\textwidth]{../../Ressourcen/Abbildungen/Nutzengröße_Bewertungseinheit_Zerlegt_Beispiel.jpg}
    \caption{Beispiel: Disaggregation nach Nutzengrößen. (Eigene Darstellung)}
    \label{fig:Disagggregation_Bilanzraum_Nutzengrößen_Beispiel}
\end{figure}


Eine Analyse der Gebäudezonen innerhalb eines Gebäudes durch Disagggregation nach Untersuchungsgegenstand wie sie in 
\eqref{fig:Disagggregation_Bilanzraum_Untersuchungsgegenstand} visualisiert ist kann zur Identifikation wesentlicher Energieeinsätze durch die Analyse und Unterscheidung 
von Energieeinsätzen innerhalb von Gebäude(-zonen) beitragen.



Abbildung \eqref{fig:Disagggregation_Bilanzraum_Untersuchungsgegenstand_Beispiel} visualisiert beispielhaft, wie eine Disagggregation des Untersuchungsgegenstands 
zur Analyse und Unterscheidung von Energieeinsätzen innerhalb eines Untersuchungsgegenstands beitragen kann.
Zur Bewertung der einzelnen Bilanzräume werden die selben Energieleistungskennzahlen wie in Abbildung \eqref{fig:Disagggregation_Bilanzraum_Nutzengrößen_Beispiel} 
genutzt, allerdings wird der Bilanzraum anhand des Untersuchungsgegenstands disaggregiert.
In diesem Beispiel macht die Gebäudezone 2 mit einem Energieeinsatz von 18.000 kWh 60\% des Gesamtenergieverbrauchs des Gebäudes aus während Gebäudezone 3 mit 
einem Energieeinsatz von 3.000 kWh nur 10\% des Gesamtenergieverbrauchs ausmacht.
Der spezifische Energieeinsatz ist in Gebäudezone 2 mit  \( 300 \,\frac{\text{kWh}}{\text{m}^2} \) am höchsten und somit ist die Energieeffizienz am geringsten. 
In Gebäudezone 3 ist mit \( 100 \,\frac{\text{kWh}}{\text{m}^2} \) der spezifische Energieeinsatz am niedrigsten und die Energieeffizienz somit am höchsten.



\begin{figure}[H]
    \centering
    \includegraphics[width=0.7\textwidth]{../../Ressourcen/Abbildungen/Untersuchungsgegenstand_Zerlegt_Beispiel.jpg}
    \caption{Beispiel: Disaggregation nach Untersuchungsgegenstand. (Eigene Darstellung)}
    \label{fig:Disagggregation_Bilanzraum_Untersuchungsgegenstand_Beispiel}
\end{figure}
\subsection{Methodik zur Erfüllung von ISO 50001 Anforderungen}

\subsubsection{Integration von relevanten Variablen}
Relevante Variablen werden von der DIN EN ISO 50001:2018-12 (Kapitel 3.4.9) als quantifizierbarer Faktor, der die energiebezogene Leistung wesentlich beeinflusst sich 
routinemäßig ändert definiert. 
Die relevanten Variablen dürfen gemäß E DIN ISO 50006:2024-07 entweder direkt gemessen oder aus Messungen abgeleitet werden (\cite[S. 18]{DIN50006.2024}).
Beispiele für relevante Variablen sind Wetterbedingungen, Betriebsbedingungen wie Innenraumtemperatur oder Lichtstärke und Arbeitsstunden (\cite[Kapitel 3.4.9]{DIN50001.2018}).
Die Integration von relevanten ist einer der Grundlegenden Anforderungen zur erfüllung von Anforderungen der DIN EN ISO 50001:2018-12.
So sollen im rahmen der Bestimmtung der EnPI-Grenzen Organisationen die für die einzelnen Grenzen relevanten Variablen bestimmen (\cite[S. 17]{DIN50006.2024}).
Für die statistische Analyse der EnPI-Werte ist es notwendig dass der Energieverbrauch und die Daten der zugehörigen relevanten Variablen 
die gleichen Zeitintervalle umfassen (\cite[S. 20]{DIN50006.2024}).
Zur ermittlung wesentlicher Energieeinsätze ist ebenfalls nicht nur die Messung deren Energieverbrauchs notwendig, sondern auch die Messung und Überwachung der 
relevanten Variablen bezüglich SEUs (\cite[S. 23]{DIN50001.2018}). 


\subsubsection{Datengetriebener Ansatz}

Die DIN EN ISO 50001:2018-12 schreibt einen Datengetriebenen Ansatz vor indem Sie von Organisationen welche die Norm umsetzen möchten fordert dass ein Plan zur 
Überwachung und Messung der Hauptmerkmale erstellt und dokumentiert wird (\cite[S. 30ff.]{DIN50001.2018}).
Um die genauen Vorgaben der DIN EN ISO 50001:2018-12 zur ermittlung von Energieeinsätzen zu formulieren, wurde die E DIN ISO 50006:2024-07 veröffentlicht,
welche sich mit der Messung der energiebezogenen Leistung im Rahmen der DIN EN ISO 50001:2018-12 befasst (\cite[S. 1]{DIN50006.2024}).

Organisationen sollen nach E DIN ISO 50006:2024-07 (Kapitel 5.1) Arten des Energieeinsatzes identifizieren und zum einen deren aktuellen, sowie früheren 
Energieverbrauch, zum anderen die aktuelle und frühere Energieeffizienz auf Basis von Messungen und anderen Daten bewerten. 
SEUs werden dann anhand der Analyse dieser Informationen, unter berücksichtigung von Faktoren die die energiebezogene Leistung beeinflussen, 
identifiziert (\cite[Kapitel 5.1]{DIN50006.2024}). 
Folglich bestimmt die Komplexität der Energiedatensammlung auch die potentielle Komplexität der Abbildung und Energiebilanzierung 
des Organisationskontext über Bilanzräume. 
Die Datengetriebenen Ermittlung von Energieeinsätzen in Bilanzräumen fordert folglich die Energiedatensammlung der Energieeinsätze, 
zu der auch die DIN EN ISO 50001:2018-12 Vorgaben macht.

Bei der Auswahl von EnPIs sollen Organisationen ihre vorhandenen Fähigkeiten zur Messung und Überwachung in Bezug auf den 
Energiverbrauch und relevante Variablen berücksichtigen (\cite[S. 21]{DIN50006.2024}).
So muss eine Organisation die Energieflüsse die eine EnPI-Grenze überschreiten messen, wobei die gemessenen Daten sowohl die 
zugelieferte als auch vor Ort erzeugte Energie berücksichtigt die die EnPI-Grenze überschreitet und gespeichert wird (\cite[S. 17]{DIN50006.2024}).
Folglich beeinflusst die Komplexität der Energiedatensammlung die Menge der abbildbaren Energieleistungskennzahlen. 
Eine Organisation soll also die sich auf den Energieverbrauch beziehenden Daten und relevanten Variablen für jede EnPI spezifizieren und erfassen (\cite[S. 18]{DIN50006.2024}).
Falls einige EnPIs aufgrund begrenzter Daten oder anderen Hürden nicht messbar sein, soll die Organisation die EnPIs bewerten und in Folge überarbeiten oder 
zusätzliche Zähler, Messungen oder Modellierungsverfahren einführen (\cite[S. 18]{DIN50006.2024}).

Die DIN EN ISO 50001:2018-12 (2018, Kapitel 6.6, A.6.6) stellt Anforderungen und Qualitätskriterien an die Datensammlung in Organisationen.
Die Norm verpflichtet Organisationen dazu, Hauptmerkmale ihrer Tätigkeiten, die sich auf die energiebezogene Leistung auswirken, zu identifizieren und diese in geplanten
Zeitabständen zu messen, zu überwachen und zu analysieren (\cite[S. 23]{DIN50001.2018}).
Teil der zu erfassenden Hauptmerkmale ist der Energieverbrauch bezüglich wesentlicher Energieeinsätze und die relevanten Variablen bezüglich SEUs (\cite[S. 23]{DIN50001.2018}). 
Die Komplexität der Umsetzung ist dabei nicht vorgeschrieben und kann von einfachen Zählwerten bis hin zu umfangreichen Werten aus Überwachungs- und Messsystemen mit
Softwareanwendung reichen (\cite[S. 36]{DIN50001.2018}).

Zusätztlich zu den Vorgaben über den Umfang der Energiedatensammlung gibt die DIN EN ISO 50001:2018-12 auch Qualitätskriterien der Energiedatensammlung vor.
So muss eine geeignete Abtastzeit der Datensammlung gewählt werden (\cite[S. 20]{DIN50006.2024}), und im Rahmen von Analysen müssen Einschränkungen der Daten 
wie Genauigkeit, Präzision und Konsistenz der Energiedatenerfassung Rechnung getragen werden (\cite[S. 37]{DIN50001.2018}).  
Da sich die DIN EN ISO 50001:2018-12 auf die Veränderung der energiebezogenen Leistung bezieht ist die Wiederholbarkeit ein wichtigeres Qualitätskriterium der 
Energiedatensammlung als die Präzision der Messung (\cite[S. 3]{Szajdzicki.2017}).



%%%%%%%%%%%%%%%%%%%%%%%%%% Konzept %%%%%%%%%%%%%%%%%%%%%%%%%%

\chapter{Konzeption und implementation in EMS-EDM Prophet®}
\section{Anforderungen an das Bilanzraumkonzept}

Dieser Abschnitt befasst sich mit den im Kapitel 2 erarbeiteten Erkenntnissen und formuliert deren Grundlage Anforderungen an das Bilanzraumkonzept.
Zur formulierung der Anforderungen werden Herausforderungen welche sich aus den Grundlagen der Energiebilanzierung und aus den betrachteten Konzepten zur Abbildung 
von Energiebilanzen ergeben betrachtet.
Des weiteren werden Anforderungen auf Grundlage der Angaben der DIN EN ISO 50001:2018-12 zu Energiemanagementsystemen in den Aspekten: 
Energieleistungskennzahlen und wesentliche Energieeinsätze erarbeitet.
Die erarbeiteten Anforderungen sind in Tabelle \eqref{tab:funktionale_anforderungen} dargestellt.


\begin{longtable}{| m{0.06\textwidth} | m{0.16\textwidth} | m{0.32\textwidth} | m{0.46\textwidth} |}
    \caption{Funktionale Anforderungen an das Bilanzraumkonzept auf Basis der theoretischen Erkenntnisse.} \\
    \label{tab:funktionale_anforderungen} \\ 
    
    \hline
    \textbf{Index} & \textbf{Aspekt} & \textbf{Anforderung} & \textbf{Begründung} \\
    \hline
    \multicolumn{4}{|c|}{\textbf{Anforderungen basierend auf den Grundlagen der Energiebilanzierung}} \\
    \hline
    1 
    & Abbildung von Energieströmen, -quellen und -senken 
    & Das Bilanzraumkonzept \textbf{muss} beliebig viele zu- und abfließende Energieströme, Energiequellen und Energiesenken abbilden. 
    & Die mathematische Beschreibung der Bilanzierung nach Ahrendts (vgl. Gleichung \eqref{BilanzierungsgleichungAhrendt}) beschreibt die Veränderung der bilanzierten Zustandsgröße in abhängigkeit dieser Komponenten. \\
    \hline
    2
    & Abbildung des Untersuchungsgegenstands 
    & Das Bilanzraumkonzept \textbf{muss} eine Beschreibung des Untersuchungsgegenstands abbilden. 
    & Der Untersuchungsgegenstand beschreibt den in der Organisation definierten Bereich der Bilanziert wird. \\
    \hline
    3
    & Zeitliche Abgrenzung 
    & Das Bilanzraumkonzept \textbf{muss} die Möglichkeit bieten, die Bilanzierung zeitlich abzugrenzen. 
    & Die Zustandsgröße der Bilanz ist nach der Mathematischen Beschreibung von Ahrendts (vgl. Gleichung \eqref{BilanzierungsgleichungAhrendt}) abhängig vom Zeitintervall, 
    in dem Energieströme, -quellen und -senken 
    wirken. \\
    \hline
    4
    & Integration von Energiespeichern 
    & Das Bilanzraumkonzept \textbf{muss} die Abbildung und Integration von energiespeichernden Komponenten ermöglichen. 
    Im zuge der Integration von energiespeichernden Komponenten \textbf{müssen} dem Energiespeicher zu- und abfließende Energieströme 
    abbilden. 
    & Energiespeicher wirken sich nach der Mathematischen Beschreibung einer Energiebilanz nach Rönsch (vgl. Gleichung \eqref{energiebilanzierungsgleichung_Rönsch}) 
    auf das Verhalten der bilanzierten Zustandsgröße aus.
    Energiespeicher nehmen Energie auf und geben sie zeitversetzt ab. \\
    \hline
    5
    & Kompatibilität mit Bewertungseinheiten 
    & Das Bilanzraumkonzept \textbf{muss} Energieströme, -quellen und -senken der Einheit kWh und deren Vielfache unabhängig von Energieformen und Energieträger in einem Bilanzraum 
    abbilden können. 
    & Die bevorzugte Bewertungseinheit für Energieformen ist kWh und deren Vielfaches (\cite[S. 65]{Konstantin.2023}).\\
    \hline
    6
    & Hierarchische Disaggregation eines Bilanzraums 
    & Das Bilanzraumkonzept \textbf{muss} die hierarchische Disaggregation nach Untersuchungsgegenstand von Bilanzräumen in Teilbilanzräume abbilden können. 
    & Nach Engelmann (2015) und Miller (2016) haben Bilanzräume die Eigenschaft der Zerlegbarkeit.\\
    \hline
    7
    & Disaggregation nach Verbrauchsarten 
    & Das Bilanzraumkonzept \textbf{soll} die Disaggregation von Bilanzräumen nach Verbrauchsarten oder Nutzengrößen abbilden können. 
    & Die Unterscheidung von Verbrauchsarten ermöglicht die Erfassung von Energiedaten einer Organisation (\cite[S. 14]{Hohnhold.2013}). \\
    \hline
    8
    & Quantifizierung von Energiedienstleistungen & 
    Das Bilanzraumkonzept \textbf{soll} in Anlehnung an das von Miller (2016) entworfene Konzept der Bewertungsräume aus dem Bilanzraum austretende Energieströme 
    in Form von Energiedienstleistungen mit einer Bewertungseinheit konkretisieren und quantifizieren. 
    & Das in der Energiewertschöpfungskette (vgl. Abbildung \eqref{fig:Energieflussschema_Posch}) visualisierte Problem:
    dass Energie in unterschiedlichen Formen auftritt und insbesondere im Rahmen der Bilanzierung von Gebäudeenergie zufließende Energieströme in Form von messbarer 
    Endenergie auftreten während die abfließenden Energieströme in Form nicht messbarer Energiedienstleistungen auftreten können wird durch die von 
    Miller (2016) beschriebene Quantifizierung von Energie in Form von Energiedienstleistungen adressiert. \\
    \hline
    9
    & Aggregation zufließender Energieströme 
    & Das Bilanzraumkonzept \textbf{soll} in Anlehnung an das von Miller (2016) entworfene Konzept der Bewertungsräume eine Aggregation mehrerer zufließender 
    Energieströme zur Erfüllung derselben Energiedienstleistung abbilden können. 
    & Mehrere einfließende Energieströme und Energiequellen können zur Deckung des Nutzenergiebedarfs der gleichen Energiedienstleistung beitragen,
    weshalb ihre Aggregation notwendig ist um Rückschlüsse auf den Energieverbrauch einer Energiedienstleistung zu ziehen. \\
    \hline
    \multicolumn{4}{|c|}{\textbf{Anforderungen basierend auf den ISO 50001 Anforderungen}} \\
    \hline
    10
    & Definition von Energieleistungskennzahlen 
    & Das Bilanzraumkonzept \textbf{muss} die Definition von Energieleistungskennzahlen (EnPI) in einem Bilanzraum ermöglichen. 
    & Energieleistungskennzahlen sind zentral für die Bewertung der energiebezogenen Leistung im Energiemanagement nach ISO 50001. \\
    \hline
    11
    & Aggregation von EnPI-Werten 
    & Ein im Bilanzraum definierter EnPI \textbf{muss} die dazugehörigen EnPI-Werten in einem Bewertungszeitraum zu einem EnPI-Wert im Bewertungszeitraum aggregieren können. 
    & Die Aggregation erlaubt eine Bewertung der energiebezogenen Leistung über einen Bewertungszeitraum hinweg und ermöglicht den Vergleich mit dem EnB zur ermittlung der Energiebezogenen Leistung 
    (vgl. Abbildung \eqref{fig:Beziehung_EnPI_EnB_ISO_50006}). \\
    \hline
    12
    & Energetische Ausgangsbasis 
    & Das Bilanzraumkonzept \textbf{muss} in einem Bilanzraum für alle definierten EnPI eine energetische Ausgangsbasis bereitstellen und im Bezugszeitraum aggregieren können.  
    & Die Differenz zwischen der Ausgangsbasis und dem EnPI-Wert bestimmt die Verbesserung der Energieeffizienz (vgl. Abbildung \eqref{fig:Beziehung_EnPI_EnB_ISO_50006}).
    Da die Berechnung einer EnB mit einem anhand eines Energiemodells berechnet wird ist die Berechnung der EnBs kein Aspekt dieser Anforderung \\
    \hline
    13
    & Definition von operativen Energiezielen 
    & Das Bilanzraumkonzept \textbf{muss} die Möglichkeit bieten, für einen definierten EnPI ein operatives Energieziel anzugeben und dessen Erfüllung zu evaluieren. 
    & Die Bewertung der Zielerreichung ist essenziell für die kontinuierliche Verbesserung der energiebezogenen Leistung (vgl. Abbildung \eqref{fig:Beziehung_EnPI_EnB_ISO_50006}). \\
    \hline
    14
    & Zeit-Werte-Verlauf von EnPI und EnB 
    & Das Bilanzraumkonzept \textbf{muss} den Zeit-Werte-Verlauf für alle im Bilanzraum definierten EnPIs und die dazugehörige EnBs ausgeben können. 
    & Eine zeitabhängige Analyse ermöglicht Erkenntnisse über das zeitabhängige Verhalten und den Einfluss relevanter Variablen (\cite[S. 14]{DIN50006.2024}). \\
    \hline
    15
    & Abbildbare Kennzahlenarten 
    & Das Bilanzraumkonzept \textbf{muss} in einem Bilanzraum EnPIs der Art: absolute Zahlen sowie Index-, Beziehungs- und Gliederungszahlen berechnen können. &
    Unterschiedliche Arten von Kennzahlen bieten unterschiedliche Perspektiven auf die Bewertung eines Bilanzraums. \\
    \hline
    16
    & Abgrenzung nach Grenzniveaus 
    & Ein Bilanzraum im Bilanzraumkonzept \textbf{muss} nach allen definierten EnPI-Grenzniveaus (vgl. Abbildung \eqref{fig:EnPI_Grenzniveaus}) abgegrenzt werden können. 
    & Die räumliche und zeitliche Abgrenzung eines Bilanzraums beeinflusst die Berechnung und Vergleichbarkeit von EnPI-Werten. \\
    \hline
    17
    & Bestimmung des Energieeinsatzes
    & Für jeden Bilanzraum im Bilanzraumkonzept \textbf{muss} der Energieeinsatz in einem Bilanzzeitraum berechnet werden können. 
    & Auf Grundlage des Vergleichs von Energieeinsätzen können wesentliche Energieeinsätze auf Grundlage der von der Organisation festgelegten Kriterien identifiziert werden. \\
    \hline
    18
    & Bestimmung des spezifischen Energieeinsatzes
    & Für jeden Bilanzraum im Bilanzraumkonzept  der den Energievebrauch einer Energiedienstleistungen bilanziert \textbf{muss} der spezifische Energieeinsatz 
    (vgl. Gleichung \eqref{SepzifischerEnergieinsatzMiller}) in einem Bilanzzeitraum berechnet werden können. 
    & Der spezifische Energieeinsatz kann als Maß der Energieeffizienz bei der Erfüllung einer Energiedienstleistung genutzt werden und somit Bilanzräume mit Potential zur 
    Energieeinsparung aufzeigen. \\
    \hline
    19
    & Auswertung von Energieeinsätzen in Disaggregierten (Teil-)Bilanzrräumen
    & Das Bilanzraumkonzept \textbf{muss} Funktionen zur Analyse des Energieeinsatzes eines Bilanzraums im Verhältnis zu seinen disaggregierten Teilbilanzräumen bereitstellen.
    & Die Analyse des Energieeinsatzes in Teilbilanzräumen im Verhältnis zum Gesamtbilanzraum ermöglicht die Erfassung des Anteils des Energieverbrauchs des 
    Teilbilanzraums am Gesamtenergieverbrauch \\
    \hline 
\end{longtable}
Quelle: Eigene Darstellung basierend auf den Erkenntnisse aus Kaptiel 2: Stand der Forschung und Theoretische Grundlagen.
% \section{Ausgangszustand: EMS-EDM Prophet®}

% \section{Anforderungen}
% \subsection{Modellierung von Bilanzräumen}

% \subsection{Abbildung von Metriken zur Bewertung von Bilanzräumen}

% \subsection{Technische Anforderungen und Datenkommunikation}


% \section{Umsetzungskonzept für EMS-EDM Prophet®}

% \subsection{Systemarchitektur}
% \subsection{Datenmodell und Datenbankdesign}
% \subsection{Technische Umsetzung und Datenkommunikation}
% \subsection{EnPI Abbildung}
% \subsection{Test- und Validierungskonzept}
% \subsection{Sicherheitskonzept}
% \subsection{Bedingungen und Anforderungen an die Laufzeitumgebung}

% \section{Technische Realisierung}
% In diesem Kapitel sollen alle technischen Aspekte der Umsetzung beschrieben werden.
% \section{Umsetzungsablauf}
% In diesem Kapitel soll der Ablauf der Umsetzung beschrieben und begründet werden.
% \section{Ergebnis}
% In diesem Kapitel soll das Ergebnis der Umsetzung beschrieben werden.
% \subsection{Anforderungsumsetzung}
% In diesem Kapitel soll beschrieben werden wie die Anforderungen umgesetzt wurden und welche Anforderungen umgesetzt wurden.
% \subsection{Vergleich zum bestehenden System}
% In diesem Kapitel soll der Vergleich zum bestehenden System gezogen werden und erläutert werden was sich verändert hat.


%%%%%%%%%%%%%%%%%%%%%%%%%% Implementation und Evaluation %%%%%%%%%%%%%%%%%%%%%%%%%%

\chapter{Evaluation}

% \section{Einleitung}
% \subsection{Ziel der Evaluation}
% \subsection{Methodik}

% \section{Metriken und Kennzahlen}
% \subsection{Definition der Erfolgsmessung}
% \subsection{Quantitative Metriken}
% \subsection{Qualitative Metriken}

% \section{Experimenteller Aufbau}
% \subsection{Beschreibung des Experimentes}
% \subsection{Testumgebung und -bedingungen}
% \subsection{Datensätze und Szenarien}

% \section{Durchführung und Ergebnisse}
% \subsection{Durchführung}
% \subsection{Quantitative Ergebnisse}
% \subsection{Qualitative Ergebnisse}

% \section{Vergleich mit alternativen Ansätzen}

% \section{Diskussion der Ergebnisse}

% \section{Zusammenfassung der Evaluation}

\chapter{Fazit}
% \section{Zusammenfassung der Ergebnisse}
% \section{Ausblick auf zukünftige Arbeiten}

\printbibliography[title={Literaturverzeichnis}]

\appendix
\chapter{Anhang}
\section*{Selbstständigkeitserklärung}

\end{document}
